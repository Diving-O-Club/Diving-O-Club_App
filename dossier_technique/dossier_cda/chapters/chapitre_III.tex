\chapter{Méthodologie et organisation}

\section{Gestion de projet avec GitHub}

\textbf{Gestion du projet et organisation du code}\\[0.2cm]

L’ensemble du projet Diving O Club est géré et suivi sur GitHub afin de garantir une traçabilité complète des décisions, du code et de l’avancement.

\begin{itemize}
  \item \textbf{Organisation du code}
  \begin{itemize}
      \item Dépôt GitHub unique structuré (frontend / backend / documentation)
      \item Branches par fonctionnalité (\texttt{feature/...}), revues avant merge sur \texttt{develop}, puis release sur \texttt{main}
      \item Messages de commit normalisés (Conventionnal Commit)
  \end{itemize}

  \item \textbf{Planification \& suivi (GitHub Projects)}
  \begin{itemize}
      \item Tableau Kanban structuré : \textit{Backlog $\rightarrow$ Next up $\rightarrow$ In Progress $\rightarrow$ In Review / Testing $\rightarrow$ Done}
      \item Découpage du projet en \textit{milestones} : Phase de conception et cadrage, POC, MVP, Bêta, v1.
      \item Issues créées pour chaque fonctionnalité, bug ou tâche technique
  \end{itemize}

  \item \textbf{Lien entre code et gestion}
  \begin{itemize}
      \item Chaque issue liée à une branche
      \item Pull requests associées aux issues et validées via checklist
      \item Documentation des décisions techniques dans \texttt{/docs}
  \end{itemize}

  \item \textbf{Qualité \& automatisation}
  \begin{itemize}
      \item GitHub Actions pour lint / tests / build
      \item Revue systématique avant fusion sur \texttt{develop}
      \item Releases taguées avec changelog versionné (\texttt{v1.0.0}, etc.)
  \end{itemize}
\end{itemize}

Cette méthode assure une traçabilité complète, un développement par incrémentation contrôlé et une preuve de professionnalisation du cycle de vie logiciel.


\section{Rituels de suivi du projet}

Afin d’assurer un pilotage rigoureux du projet Diving O Club, plusieurs rituels de suivi ont été mis en place. 
Bien que le projet soit développé individuellement, l’organisation adoptée repose sur des \textbf{cycles courts}, inspirés du cadre agile Scrum, ce qui permet un développement itératif, incrémental et facilement ajustable selon les retours du club partenaire.

\begin{itemize}

\item \textbf{Sprints d’une semaine (cycle court agile)}
Chaque semaine constitue un sprint avec un objectif clair (3 à 5 tâches maximum).  
Cette durée courte permet de livrer des incréments réguliers du produit, de limiter les risques et de faciliter la priorisation continue.  
Une auto-revue de sprint est réalisée en fin de cycle afin d’évaluer l’atteinte des objectifs, ajuster le backlog et préparer le sprint suivant.

\item \textbf{Revue hebdomadaire (Weekly Review)}
Analyse de l’avancement, mise à jour du tableau Kanban GitHub, clôture des issues terminées et identification des blocages éventuels.  
Ce rituel assure un recalibrage permanent de la charge, de la qualité du code et des priorités métier.

\item \textbf{Tests exploratoires mensuels (avec 3 adhérents, 1 moniteur, 1 membre du comité)}
Chaque mois, un test terrain est réalisé sur les parcours clés (inscription, paiement, certificat).  
Les retours, blocages et améliorations sont intégrés au backlog sous forme d’issues, permettant un apprentissage continu et une amélioration progressive de l’expérience utilisateur.

\end{itemize}

Ces rituels garantissent un développement structuré et agile, fondé sur des boucles courtes de feedback, une amélioration continue et une forte capacité d’adaptation aux besoins réels du club testeur.


\subsection{User Stories et estimation de temps}


\textbf{User stories avec estimations :}


Dans une logique inspirée des méthodes agiles, le périmètre fonctionnel du projet a été 
structuré en trois niveaux : \textbf{Épics}, \textbf{User Stories} et \textbf{Tâches techniques}. 

\begin{itemize}
  \item Les \textbf{Épics} regroupent les grands blocs fonctionnels du projet (ex. Authentification, Événements, Paiements).
  \item Chaque \textbf{User Story} exprime un besoin utilisateur clair, formulé selon le format 
  \textit{« En tant que [rôle], je veux [objectif], afin de [bénéfice] »}.
  \item Les \textbf{Tâches} découlent directement des User Stories et correspondent aux actions techniques 
  spécifiques nécessaires à leur implémentation.
\end{itemize}

Cette structuration apporte plusieurs avantages : une meilleure lisibilité du périmètre, 
une priorisation facilitée, une estimation du temps plus fiable et une traçabilité directe 
entre besoins métier et code. L’intégralité du backlog est maintenue dans GitHub Projects 
pour assurer un suivi précis et incrémental.

\vspace{0.5cm}
\noindent
\textbf{Exemple complet : EP-01 — Authentification \& Comptes}

\paragraph{Épic EP-01 — Authentification \& Comptes}  
Assurer la création, la validation et la gestion sécurisée des comptes utilisateur, incluant 
la connexion, la gestion des sessions et la récupération de mot de passe.

\subsubsection*{User Stories associées}

\textbf{US-01 — Création de compte}  
\textit{En tant qu'utilisateur, je veux créer un compte sécurisé afin d'accéder à mon espace personnel.}  
\textbf{Critères d’acceptation :}  
\begin{itemize}
  \item Validation du format email et du mot de passe.
  \item Compte créé en base, rôle par défaut = \texttt{user}.
  \item Redirection vers l’espace personnel si inscription réussie.
\end{itemize}

\textbf{US-02 — Connexion / Déconnexion}  
\textit{En tant qu'utilisateur, je veux me connecter avec email et mot de passe afin d’accéder à mon tableau de bord.}  
\textbf{Critères d’acceptation :}  
\begin{itemize}
  \item JWT d’accès court + cookie refresh httpOnly.
  \item Message générique en cas d’échec (anti-fuite d’information).
  \item Déconnexion invalide le refresh token.
\end{itemize}

\textbf{US-03 — Gestion du rôle utilisateur}  
\textit{En tant que membre du comité, je veux modifier le rôle d’un utilisateur afin d’adapter ses permissions.}  
\textbf{Critères d’acceptation :}  
\begin{itemize}
  \item Mise à jour immédiate du rôle.
  \item Action tracée dans les logs d’audit.
\end{itemize}

\textbf{US-04 — Récupération de mot de passe}  
\textit{En tant qu'utilisateur, je veux récupérer mon accès en cas d’oubli de mot de passe.}  
\textbf{Critères d’acceptation :}  
\begin{itemize}
  \item Envoi d’un lien signé avec expiration.
  \item Réinitialisation invalide toutes les sessions précédentes.
\end{itemize}

\vspace{0.4cm}

\subsubsection*{Tâches techniques associées}

\begin{itemize}
  \item Mise en place du module \texttt{AuthModule} dans Nest.js.
  \item Implémentation des DTOs (login, register, refresh, reset).
  \item Hashage des mots de passe avec bcrypt.
  \item Génération JWT (access + refresh).
  \item Cookies sécurisés \texttt{httpOnly}, \texttt{SameSite=Lax}.
  \item Middleware de validation du token.
  \item Route de réinitialisation du mot de passe + email signé.
  \item Guard RBAC pour protéger les routes sensibles.
  \item Ensemble de tests API : login, refresh, register, reset.
\end{itemize}

\vspace{0.3cm}

Cet exemple illustre la logique appliquée à l’ensemble des autres Épics.  
Le backlog complet contient plusieurs dizaines de User Stories et tâches ; afin d’éviter une
surabondance dans le document, seul cet exemple est détaillé ici.
Les autres éléments sont structurés et maintenus dans l’outil de gestion de projet.
\vspace{0.5cm}


\textbf{Colonnes du tableau Kanban :}
\begin{itemize}
    \item \textbf{Backlog :} Fonctionnalités à développer
    \item \textbf{To Do :} Tâches prêtes pour le sprint
    \item \textbf{In Progress :} Tâches en cours (WIP limit: 3)
    \item \textbf{Review :} Code en attente de validation
    \item \textbf{Done :} Fonctionnalités livrées
\end{itemize}


\section{Versioning GitHub et conventions}

\textbf{Stratégie de gestion Git — Git Flow adapté solo}\\[0.2cm]

\textbf{Modèle retenu : Git Flow (adapté solo)}\\[-0.1cm]
\begin{itemize}
  \item \textbf{Branches permanentes :} 
  \begin{itemize}
    \item \texttt{main} — version stable, déployée
    \item \texttt{develop} — intégration
  \end{itemize}
  
  \item \textbf{Branches temporaires :}
  \begin{itemize}
    \item \texttt{feature/<scope>} — nouvelle fonctionnalité
    \item \texttt{fix/<scope>} — correctif non critique
    \item \texttt{hotfix/<scope>} — correctif critique depuis \texttt{main}
    \item \texttt{release/<version>}
  \end{itemize}
\end{itemize}

\textbf{Stratégie de merge}\\[-0.1cm]
\begin{itemize}
  \item Vers \texttt{develop} : pull request (relecture, CI OK), \textbf{squash \& merge}
  \item Vers \texttt{main} : via \texttt{release/} après tests finaux
  \item Hotfix : branche depuis \texttt{main} -> PR vers \texttt{main} + cherry-pick vers \texttt{develop}
\end{itemize}

\textbf{Nommage des branches — exemples}\\[-0.1cm]
\begin{itemize}
  \item \texttt{feature/auth-login}, \texttt{feature/events-calendar}, \texttt{fix/payment-status}
  \item \texttt{release/v1.0.0}, \texttt{hotfix/jwt-expiry}
\end{itemize}

\textbf{Conventions de commit (Conventional Commits)}\\[-0.1cm]
\begin{itemize}
  \item \texttt{feat: auth avec JWT + refresh} \textit{Closes \#12}
  \item \texttt{fix: corrige statut paiement HelloAsso} \textit{Closes \#34}
  \item \texttt{docs: ajoute README déploiement} \textit{Closes \#7}
  \item \texttt{test: ajoute tests API inscriptions} \textit{Closes \#21}
  \item \texttt{refactor: isole service certificats} \textit{Closes \#18}
  \item \texttt{chore: met à jour dépendances} \textit{Closes \#3}
  \item \texttt{perf: optimise requête liste événements} \textit{Closes \#27}
\end{itemize}

\vspace{0.3cm}

\noindent
Chaque commit référence explicitement une issue via \texttt{Closes \#ID}, 
ce qui permet :
\begin{itemize}
  \item la fermeture automatique de l’issue une fois la PR fusionnée ;
  \item une traçabilité parfaite entre le code livré et les besoins exprimés ;
  \item un historique clair pour le suivi du projet dans GitHub.
\end{itemize}

\vspace{0.5cm}

\textbf{Pull Requests (PR)}\\[-0.1cm]
\begin{itemize}
  \item Template PR incluant checklist :
  \begin{itemize}
    \item Issue liée (\textit{Closes \#12})
    \item Tests ajoutés / mis à jour
    \item Lint \& CI OK
    \item Impact sécurité (auth / rôles) vérifié
  \end{itemize}
  \item Règles : pas de commit direct sur \texttt{main}, 1 PR = 1 sujet, captures UI si besoin
\end{itemize}

\vspace{0.5cm}

\textbf{Protection des branches}\\[-0.1cm]
\begin{itemize}
  \item \texttt{main} : protégée (CI obligatoire, squash \& merge uniquement, review requise)
  \item \texttt{develop} : CI obligatoire, pas de push direct en production
\end{itemize}

\vspace{0.5cm}

\textbf{Versioning \& releases}\\[-0.1cm]
\begin{itemize}
  \item SemVer \texttt{MAJOR.MINOR.PATCH}
  \item \texttt{v1.0.0} = MVP (juin 2026)
  \item \texttt{v1.1.0} = dashboard + notifications
  \item \texttt{v1.2.0} = matériel + carnet de plongée
  \item Changelog généré depuis commits (\texttt{feat}/\texttt{fix}) — \texttt{CHANGELOG.md}
  \item Tags sur \texttt{main} lors des releases
\end{itemize}

\vspace{0.5cm}

\textbf{Traçabilité GitHub}\\[-0.1cm]
\begin{itemize}
  \item Chaque US = issue (labels : feature, bug, security, docs, P1–P3)
  \item Branche issue-based : \texttt{feature/123-events-calendar}
  \item Commits \& PR référencent l’issue (\texttt{fixes \#123})
  \item Milestones alignées avec roadmap (MVP, v1.1, v1.2)
\end{itemize}

\vspace{0.5cm}

\textbf{CI/CD (qualité)}\\[-0.1cm]
\begin{itemize}
  \item GitHub Actions : \texttt{lint -> test -> build} sur PR ; déploiement sur tag \texttt{v*}
  \item Tests sur endpoints critiques (auth, inscriptions, paiements, certificats)
  \item Scan basique de vulnérabilités
\end{itemize}

\vspace{0.5cm}

\textbf{Gestion des conflits \& imprévus}\\[-0.1cm]
\begin{itemize}
  \item Résolution locale puis rebase avant PR
  \item Urgences via \texttt{hotfix/}, patch immédiat, sync vers \texttt{develop}
\end{itemize}

\vspace{0.5cm}

\textbf{Documentation}\\[-0.1cm]
\begin{itemize}
  \item \texttt{README.md}
  \item User stories
\end{itemize}


\section{Planification et outils de suivi}

\textbf{Planification et pilotage du projet}\\[0.2cm]

La planification du projet repose sur une approche duale :\\[-0.1cm]
\begin{itemize}
  \item \textbf{Roadmap GitHub} pour la vision macro (jalons, dépendances, versions)
  \item \textbf{Kanban GitHub Projects} pour le suivi opérationnel au quotidien (User Stories, issues, flux de production)
\end{itemize}

Cette organisation permet d’allier \textbf{vision stratégique long terme} et \textbf{exécution incrémentale contrôlée}.

\vspace{0.3cm}

\textbf{Roadmap GitHub — Vision produit}\\[-0.1cm]
\begin{itemize}
  \item Découpage par versions :
  \begin{itemize}
  \item \texttt{MVP} = avril 2026
  \item \texttt{v1.0} = rentrée clubs (fin août 2026)
  \item \texttt{v1.1} = dashboard + notifications (sept. 2026)
  \item \texttt{v1.2} = matériel + carnet de plongée (déc. 2026)
  \end{itemize}

  \item \textbf{Milestones} rattachés à chaque version (issues associées)
  \item Dépendances visibles : backend -> frontend -> intégration -> déploiement
  \item Gestion des risques : intégration HelloAsso, sécurité, performance
  \item Communication claire avec les parties prenantes (club testeur)
\end{itemize}

\vspace{0.3cm}

\textbf{Kanban GitHub Projects — Suivi opérationnel}\\[-0.1cm]
\begin{itemize}
  \item Colonnes : \textit{Backlog -> To Do -> In Progress -> Review -> Done}
  \item Limite WIP : 2–3 tâches maximum simultanément (évite la dispersion)
  \item 1 issue = 1 User Story (estimée, assignée, reliée à un milestone)

  \item Automatisations :
  \begin{itemize}
    \item Passage automatique de l’issue lorsque la PR est fusionnée
    \item Fermeture automatique via \texttt{Closes \#XX}
  \end{itemize}

  \item PR obligatoires pour fusionner sur \texttt{develop} ou \texttt{main}
  \item Indicateurs suivis : lead time, throughput, cycle time
\end{itemize}



\section{Estimation de temps et planification}

\textbf{Estimation du temps et faisabilité}\\[0.2cm]

L’estimation du temps est basée sur le périmètre MVP (v1.0) composé des dix User Stories essentielles.  
Le développement est planifié en alternance (\(\approx\) 3 jours / semaine en entreprise), soit environ \(\approx\) 12 jours-homme par mois.

\vspace{0.4cm}

\textbf{Estimation par fonctionnalité (MVP)}\\[0.2cm]

\begin{table}[h!]
\centering
\begin{tabular}{|p{7cm}|c|}
\hline
\textbf{Bloc fonctionnel} & \textbf{Estimation} \\ \hline
Authentification \& rôles (US-01 / US-02) & 3,5 jours \\ \hline
Calendrier \& gestion d’événements (US-03 / US-04) & 5 jours \\ \hline
Inscriptions \& annulations (US-05 / US-06) & 3,5 jours \\ \hline
Paiements HelloAsso (US-07 / US-08) & 4,5 jours \\ \hline
Certificats médicaux (US-09 / US-10) & 3,5 jours \\ \hline
\textbf{Total MVP} & \textbf{20–22 jours} \\ \hline
\end{tabular}
\end{table}


En rythme alternance : \(\approx\) 2 mois calendrier (3 jours / semaine).  
Ce planning s’inscrit dans la phase prévue : \textbf{Décembre -> Avril — Développement back-end / front-end}.

\vspace{0.4cm}

\textbf{Estimation par phase projet}\\[0.2cm]

\begin{table}[h!]
\centering
\begin{tabular}{|p{4.2cm}|p{3.2cm}|p{6cm}|c|}
\hline
\textbf{Phase} & \textbf{Période} & \textbf{Objectif} & \textbf{Charge} \\ \hline
Conception \& cadrage & Oct–Déc 2025 & UML, MCD/MLD, cahier des charges, maquettes & \(\sim\) 10 j \\ \hline
POC (stack) & Déc 2025 & Validation Next.js / Nest.js / PostgreSQL + MongoDB & \(\sim\) 3 j \\ \hline
Dév. back-end (MVP) & Jan–Mar 2026 & API, Auth, Events, Inscriptions, Certificats, Paiements (HelloAsso) & \(\sim\) 22 j \\ \hline
\textbf{MVP} & \textbf{Avr 2026} & \textbf{Version utilisable en conditions réelles (club)} & -- \\ \hline
Front + intégrations & Mai–Juin 2026 & UI Next.js/React, intégration, tests, perf & \(\sim\) 15 j \\ \hline
Stabilisation \& bêta & Juil 2026 & Corrections, retours club pilote (Aquaclub21) & \(\sim\) 8 j \\ \hline
\textbf{V1 (rentrée clubs)} & \textbf{Fin août 2026} & Déploiement final \& documentation & -- \\ \hline
\end{tabular}
\end{table}

Ce volume est cohérent avec la durée d’alternance (9 mois effectifs).

\vspace{0.4cm}

\textbf{Validation de la faisabilité}\\[0.2cm]

\begin{itemize}
  \item Le périmètre MVP a été réduit au module cœur pour rester réalisable.
  \item Les fonctionnalités secondaires sont planifiées en v1.1 / v1.2 (après retours terrain).
  \item Le planning est aligné sur la roadmap produit et les deadlines CDA.
\end{itemize}





\section{Liens utiles}

\begin{itemize}
    \item GitHub Flow/PRs: \url{https://docs.github.com/pull-requests}
    \item Git Flow: \url{https://bit.ly/gitflow-atlassian}
    \item GitHub Projects: \url{https://bit.ly/github-projects}
    \item GitHub Roadmap: \url{https://bit.ly/github-roadmap}
    \item GitHub Milestones: \url{https://bit.ly/github-milestones}
    \item User Stories: \url{https://www.mountaingoatsoftware.com/agile/user-stories}
    \item Estimation de temps: \url{https://bit.ly/time-estimation}
\end{itemize}

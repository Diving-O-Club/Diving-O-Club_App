\chapter{Sécurité applicative et RGPD}


\subsection*{6.1 Protection contre les vulnérabilités OWASP (Top 10 : 2025)}

La sécurité de l'application \textit{Diving O Club} s'appuie sur les recommandations du
\textbf{OWASP Top 10 : 2025}. Cette section présente les vulnérabilités ciblées, les mesures
de protection intégrées et la gestion sécurisée des flux entre les tiers. L'objectif est de
réduire la surface d'attaque, d'assurer l'intégrité des données et de garantir un
fonctionnement résilient même face à des comportements malveillants.

\subsubsection*{6.1.1 Vulnérabilités OWASP ciblées}

Les protections implémentées adressent spécifiquement les 10 catégories du OWASP 2025 :

\begin{itemize}
  \item \textbf{A01:2025 — Broken Access Control} : contrôles d'accès stricts via les
  \textit{Guards} Nest.js, isolation multi-tenant stricte sur le \texttt{clubId} et restrictions
  basées sur les rôles.
  
  \item \textbf{A02:2025 — Security Misconfiguration} :
  configuration sécurisée de Nest.js, désactivation des messages d'erreur internes,
  utilisation de \texttt{Helmet}, et configuration stricte des CORS.
  
  \item \textbf{A03:2025 — Software Supply Chain Failures} :
  gestion stricte des dépendances npm, analyse des vulnérabilités via \texttt{npm audit},
  mises à jour régulières, et vérification de l'intégrité des packages.
  
  \item \textbf{A04:2025 — Cryptographic Failures} :
  chiffrement des mots de passe avec Argon2, chiffrement TLS pour toutes les communications
  et gestion sécurisée des tokens JWT.
  
  \item \textbf{A05:2025 — Injection} :
  utilisation systématique de Prisma pour prévenir les injections SQL, absence de concaténation
  de chaînes dans les requêtes, et sanitisation des entrées utilisateur.
  
  \item \textbf{A06:2025 — Insecure Design} :
  architecture en couches (3 tiers), séparation stricte des responsabilités, principe du
  moindre privilège et absence de logique métier côté frontend.
  
  \item \textbf{A07:2025 — Authentication Failures} :
  implémentation de tokens courts, rotation des refresh tokens, contrôle strict des accès
  et gestion des erreurs d’authentification.
  
  \item \textbf{A08:2025 — Software or Data Integrity Failures} :
  migrations Prisma contrôlées, absence de mise à jour dynamique de scripts côté client,
  et vérifications d'intégrité lors des opérations sensibles.
  
  \item \textbf{A09:2025 — Logging \& Alerting Failures} :
  traçabilité complète des actions critiques, logs structurés, audit des événements majeurs
  (connexion, création d'événement, modification de rôle) et surveillance centralisée.
  
  \item \textbf{A10:2025 — Mishandling of Exceptional Conditions} :
  gestion unifiée des exceptions Nest.js, masquage des erreurs internes et réponses
  standardisées entre les différents tiers.
\end{itemize}

\subsubsection*{6.1.2 Mesures de protection mises en place}

Pour se prémunir contre ces vulnérabilités, plusieurs mesures défensives ont été intégrées :

\begin{itemize}
  \item \textbf{Helmet} pour configurer les headers de sécurité (HSTS, CSP, XSS-Protection).
  \item \textbf{CORS strict} afin de limiter les appels cross-origin non autorisés.
  \item \textbf{Rate limiting} pour limiter les attaques par brute force ou DoS.
  \item \textbf{Validation systématique des entrées} via DTO + \texttt{ValidationPipe}.
  \item \textbf{Sanitisation des données} pour prévenir les injections et attaques XSS.
  \item \textbf{Contrôles d'accès exhaustifs} avec \textit{AuthGuard} et \textit{RolesGuard}.
  \item \textbf{Restriction de surface d'exposition} : aucune route sensible sans authentification.
  \item \textbf{Sécurisation du transport} : API disponible uniquement via HTTPS.
\end{itemize}

\subsubsection*{6.1.3 Validation des entrées et gestion des erreurs}

La validation et la gestion des erreurs jouent un rôle essentiel dans la protection contre les
vulnérabilités OWASP :

\begin{itemize}
  \item \textbf{Validation} : les données entrantes sont vérifiées selon les DTO Nest.js,
  empêchant données incorrectes ou malveillantes d'atteindre la logique métier.
  \item \textbf{Gestion des exceptions} : Nest.js centralise la gestion des erreurs avec des
  classes telles que \texttt{BadRequestException}, \texttt{UnauthorizedException},
  \texttt{ForbiddenException}, ou \texttt{InternalServerErrorException}.
  \item \textbf{Masquage des informations internes} : aucune stack trace ni détail sensible
  n’est renvoyé au client.
  \item \textbf{Standardisation des réponses} : structure cohérente des erreurs pour
  faciliter l’intégration frontend.
\end{itemize}

\subsubsection*{6.1.4 Sécurisation des flux entre tiers}

Les communications entre frontend, backend et base de données sont strictement protégées :

\begin{itemize}
  \item \textbf{Frontend $\rightarrow$ Backend} : communication en HTTPS, utilisation du header
  \texttt{Authorization: Bearer <token>} et absence de données sensibles dans les URLs.
  \item \textbf{Backend $\rightarrow$ Base de données} : Prisma garantit des requêtes paramétrées et
  une isolation transactionnelle forte.
  \item \textbf{Éviction des données inutiles} : le backend ne retourne que les données
  strictement nécessaires (principe de minimisation).
\end{itemize}

\subsubsection*{6.1.5 Scalabilité et protection contre les attaques DoS}

L’architecture backend étant entièrement \textbf{stateless}, l'application peut facilement
monter en charge :

\begin{itemize}
  \item \textbf{Stateless JWT} : ne nécessite aucune session serveur, ce qui permet de déployer
  plusieurs instances de Nest.js derrière un load balancer.
  \item \textbf{Rate limiting} : empêche la surcharge du serveur par un trop grand nombre de
  requêtes provenant d’une même source.
  \item \textbf{Séparation des tiers} : chaque couche peut être scalée indépendamment.
\end{itemize}

Ces protections combinées assurent que \textit{Diving O Club} reste conforme aux bonnes
pratiques de sécurité en vigueur pour l'année 2025 et qu’elle résiste efficacement aux
vulnérabilités identifiées par l’OWASP.


\subsection*{6.2 Authentification et autorisation}

L'application \textit{Diving O Club} utilise un système d'authentification moderne basé sur
des tokens JWT, conforme aux bonnes pratiques de l'OWASP 2025 (A07~Authentication~Failures).
Cette section décrit les mécanismes de connexion, la gestion des mots de passe, les rôles
et permissions, ainsi que les protections associées.

\subsubsection*{6.2.1 Authentification JWT}

L'authentification repose sur un jeton d'accès (\textit{access token}) à courte durée de vie,
complété par un \textit{refresh token} permettant d'obtenir un nouveau jeton sans
redemander les identifiants. Les tokens sont générés et vérifiés côté backend via les
services Nest.js.

\begin{itemize}
  \item \textbf{Access token court} : durée limitée afin de réduire l'impact d’un vol de jeton.
  \item \textbf{Refresh token} : stocké de manière sécurisée et régénéré à chaque renouvellement.
  \item \textbf{Transport sécurisé} : les tokens sont envoyés dans l'en-tête 
  \texttt{Authorization: Bearer <token>}.
  \item \textbf{Déconnexion sécurisée} : invalidation du refresh token lors de la déconnexion.
\end{itemize}

\subsubsection*{6.2.2 Sécurisation des mots de passe}

Pour protéger les comptes utilisateur, l'application utilise \textbf{Argon2}, un algorithme
de hachage moderne spécialement conçu pour résister aux attaques par force brute ou GPU.
Aucun mot de passe n’est jamais stocké en clair en base de données.

\begin{itemize}
  \item \textbf{Argon2id} pour le hachage.
  \item \textbf{Salage automatique} intégré.
  \item \textbf{Politique de réauthentification} pour les actions sensibles.
\end{itemize}

\subsubsection*{6.2.3 Gestion des rôles et permissions}

L’application implémente un contrôle d'accès basé sur les rôles (RBAC).  
Chaque utilisateur appartient à un club et possède un rôle déterminant ses autorisations :

\begin{itemize}
  \item \textbf{admin} : gestion du club, des membres, des événements.
  \item \textbf{coach} : gestion des événements et des inscriptions.
  \item \textbf{member} : consultation et inscription aux activités.
\end{itemize}

L’autorisation est appliquée via les \texttt{Guards} Nest.js :

\begin{itemize}
  \item \textbf{AuthGuard} : vérifie la présence d’un JWT valide.
  \item \textbf{RolesGuard} : vérifie que le rôle de l'utilisateur autorise l'opération demandée.
  \item \textbf{Contrôle multi-tenant} : l'utilisateur ne peut agir que dans son propre club.
\end{itemize}

\subsubsection*{6.2.4 Gestion des erreurs d’authentification}

Les erreurs d’authentification et d’autorisation sont traitées de manière centralisée afin
d’éviter les fuites d’informations sensibles et d'assurer une cohérence entre les tiers.

\begin{itemize}
  \item \texttt{UnauthorizedException} pour les tokens invalides ou expirés.
  \item \texttt{ForbiddenException} pour les actions non autorisées.
  \item \textbf{Messages d’erreur neutres} afin de ne pas révéler si un compte existe ou non.
  \item \textbf{Journalisation des tentatives} : tentative de connexion, changement de mot de passe.
\end{itemize}

Ces mécanismes garantissent une authentification robuste, granulaire et conforme aux normes
modernes de sécurité pour l'année 2025.

\subsection*{6.3 Conformité RGPD}

L’application \textit{Diving O Club} traite des données personnelles appartenant aux membres
des clubs (identité, informations de contact, certificats de plongée, inscriptions aux
événements). Conformément au Règlement Général sur la Protection des Données (RGPD), des
mesures techniques et organisationnelles ont été mises en place pour garantir la licéité, la
sécurité et la transparence du traitement.

\subsubsection*{6.3.1 Registre des traitements}

Un registre décrit l’ensemble des traitements réalisés par l’application :  
collecte, finalité, base légale, durée de conservation et catégories de données traitées.  
Les principales finalités sont :

\begin{itemize}
  \item gestion des comptes utilisateur,
  \item gestion des membres d’un club,
  \item gestion des événements et inscriptions,
  \item suivi des qualifications et certificats.
\end{itemize}

La base légale retenue est l’\textbf{intérêt légitime du club} pour la gestion de ses activités.

\subsubsection*{6.3.2 Droits RGPD des utilisateurs}

L’application implémente les principaux droits prévus par le RGPD :

\begin{itemize}
  \item \textbf{Droit d’accès} : chaque utilisateur peut consulter ses informations personnelles.
  \item \textbf{Droit de rectification} : possibilité de mettre à jour ses données.
  \item \textbf{Droit d’effacement} : suppression du compte sur demande.
  \item \textbf{Droit à la portabilité} : export des données disponibles dans un format
        structuré.
  \item \textbf{Droit d’opposition} : possibilité de demander la suppression ou l’arrêt du traitement.
\end{itemize}

Ces actions sont réalisées via des routes sécurisées nécessitant une authentification valide.

\subsubsection*{6.3.3 Sécurité des données personnelles}

Les données personnelles sont protégées par des mécanismes techniques conformes au RGPD :

\begin{itemize}
  \item \textbf{Chiffrement en transit} via HTTPS.
  \item \textbf{Hachage Argon2} des mots de passe stockés en base.
  \item \textbf{Minimisation des données} : seules les informations nécessaires sont conservées.
  \item \textbf{Séparation des rôles} pour limiter l’accès aux données sensibles.
  \item \textbf{Logs d’accès et d’actions critiques} pour assurer la traçabilité (création
        d’événement, modification de rôle, mise à jour de profil).
\end{itemize}

\subsubsection*{6.3.4 Conservation et suppression des données}

Les données personnelles sont conservées uniquement pour la durée nécessaire à la gestion du
club. Lorsqu’un utilisateur quitte un club ou demande la suppression de ses données :

\begin{itemize}
  \item son compte est supprimé,
  \item ses inscriptions et données non essentielles sont anonymisées,
  \item les logs de sécurité sont conservés pour la durée légale en les rendant non identifiables.
\end{itemize}

\subsubsection*{6.3.5 Notification des violations}

En cas de violation de données personnelles :

\begin{itemize}
  \item un rapport d’incident est généré,
  \item les administrateurs du club sont notifiés,
  \item une notification peut être transmise à la CNIL si nécessaire.
\end{itemize}

L’objectif est de garantir transparence et réactivité conformément au RGPD.

Cette approche assure que \textit{Diving O Club} respecte les obligations légales relatives à la
protection des données, en combinant sécurité, transparence et contrôle utilisateur.

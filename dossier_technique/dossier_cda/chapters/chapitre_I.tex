\chapter{Présentation personnelle et du projet}

\section{Rôle du candidat et contexte}

\paragraph{Mon rôle :}En tant que concepteur-développeur full-stack, mon rôle consiste à identifier et analyser les besoins afin d’apporter des solutions numériques concrètes et adaptées aux problématiques rencontrées. Mes expériences précédentes, à la fois en formation et en milieu professionnel, m’ont permis d’acquérir un éventail de compétences suffisamment large pour intervenir efficacement sur l’ensemble du cycle de développement. Je suis en autonomie sur l’architecture, la conception, les choix techniques et le développement, tout en tenant compte des retours des parties prenantes du club test "Aquaclub 21".

\vspace{0.3cm}

\paragraph{Contexte organisationnel :}Je suis membre actif d’un club associatif de plongée affilié à la FFESSM depuis quatre ans et j’ai intégré le comité directeur. J’interviens sur la communication, la gestion du site internet et la coordination numérique du club. Comme dans de nombreuses structures associatives, l’organisation repose essentiellement sur le bénévolat, avec des ressources humaines et techniques limitées. Cela implique une forte autonomie dans la gestion des outils numériques pour les membres du comité.

Aujourd’hui, la gestion du club s’appuie sur une combinaison d’outils dispersés : VPDive, Google Drive, formulaires externes, tableurs partagés, messageries, documents papier. Cette fragmentation entraîne des manipulations manuelles répétitives, une perte d’informations, un risque d’erreur élevé et une charge administrative importante. Les retours du comité montrent une perte de 1 à 3 heures par semaine pour les bénévoles chargés du suivi des adhésions, certificats médicaux, inscriptions et paiements.

Un autre élément clé du contexte métier est le niveau d’aisance numérique hétérogène des pratiquants. Dans le club test Aquaclub 21, 64{,}6\% ont plus de 46 ans, dont 30\% ont plus de 61 ans. Ils peuvent facilement se retrouver en difficulté face à des interfaces complexes ou à des processus éclatés. Ils privilégient la simplicité, la lisibilité et l’automatisation.

Cette réalité accroît la nécessité d’un outil centralisé, clair et accessible, permettant de réduire les frictions, d’éviter les erreurs et de rendre les démarches (adhésion, certificats, inscriptions) plus intuitives. En tant que membre du comité, je suis régulièrement confronté à ces difficultés, ce qui m’a permis d’identifier précisément les points de friction et les besoins réels des utilisateurs.

Face à ces contraintes, les échanges avec le comité ont permis d’identifier le besoin d’une plateforme unifiée, robuste et adaptée aux processus du club associatif. L’objectif est de :

\begin{itemize}
    \item centraliser les informations
    \item améliorer l'expérience des membres, y compris les moins à l’aise avec le numérique
    \item simplifier les démarches administratives
    \item automatiser les tâches récurrentes
    \item renforcer la fluidité des processus internes en développant des fonctionnalités adaptées
\end{itemize}

Le besoin métier principal consiste donc à mettre en place un outil simple, cohérent et efficace, capable d’accompagner la gestion quotidienne du club dans un environnement associatif exigeant mais limité en ressources.

\vspace{0.5cm}

\subsection{Lien avec les compétences du référentiel CDA}

Le projet \textit{Diving O Club} constitue une mise en situation professionnelle complète permettant de mobiliser l’ensemble des compétences attendues du titre \textit{Concepteur Développeur d’Applications} (RNCP).

Dans ce cadre, mon rôle couvre les blocs de compétences suivants :
\begin{itemize}
    \item \textbf{Analyse des besoins et cadrage du projet} : recueil des besoins réels des clubs de plongée, analyse des processus administratifs existants, identification des contraintes métier et utilisateurs.
    \item \textbf{Conception de la solution applicative} : modélisation fonctionnelle et technique (diagrammes UML, schémas d’architecture, modélisation des données) afin de structurer une solution adaptée aux usages.
    \item \textbf{Développement de l’application} : conception et implémentation d’une application web full-stack, incluant les interfaces utilisateur, la logique métier et les interactions avec la base de données.
    \item \textbf{Sécurité et protection des données} : prise en compte des enjeux de sécurité applicative, d’authentification, de gestion des accès et de conformité au RGPD.
    \item \textbf{Déploiement, tests et maintenance} : mise en place d’un environnement de déploiement, d’une chaîne CI/CD, de tests de qualité et d’une démarche d’amélioration continue.
\end{itemize}

À travers ce projet, je mets en œuvre de façon concrète et cohérente les compétences professionnelles attendues d’un concepteur développeur d’applications.

\vspace{0.5cm}
\textbf{Durée et planning :} J'ai démarré mon alternance le 1\textsuperscript{er} septembre 2025, à raison de trois jours en entreprise (lundi, mardi, mercredi) et deux jours en formation (jeudi et vendredi). Ce dossier a été rédigé durant la phase de conception du projet, de septembre à décembre 2025.


\vspace{0.5cm}


\textbf{Planning du projet \textit{Diving O Club} :}

\begin{table}[h!]
\centering
\begin{tabular}{|>{\centering\arraybackslash}m{3.2cm}|>{\centering\arraybackslash}m{11cm}|}
\hline
\textbf{Période} & \textbf{Objectifs clés} \\ \hline

25 sept. --- 20 déc. 2025 & \textbf{Conception et cadrage} : deux phases d'idéation, analyse des besoins, modèles UML, MCD/MLD, wireframes, maquettes, backlog, choix techniques. \\ \hline

8 janv. --- 17 janv. 2026 & \textbf{POC} : validation de l’architecture (Next.js / Nest.js / PostgreSQL), mise en place de l'environnement Docker, affichage d’un club test avec membres et événements seedés. \\ \hline

22 janv. --- 14 fév. 2026 & \textbf{MVP} : création de compte, connexion, recherche du club test, consultation des membres et événements. \\ \hline

19 fév. --- 25 avril 2026 & \textbf{Version Bêta} : profils utilisateurs, certificats médicaux, calendrier, inscriptions, rôles (admin/moniteur/adhérent), gestion des membres et événements. \\ \hline

30 avril --- 6 juin 2026 & \textbf{Version 1} : multi-clubs, création et validation de club, invitations par email, demandes d’adhésion, paiements HelloAsso, tableaux de bord. \\ \hline

\end{tabular}
\end{table}

\textbf{QQOQCP :}

\vspace{0.2cm}

\begin{itemize}
    \item \textbf{Quoi :} Application web unifiée permettant de gérer un club de plongée associatif : adhérents, certificats médicaux, événements, inscriptions, rôles, paiements et administration.
    \item \textbf{Qui :} Clubs FFESSM et associations subaquatiques reposant majoritairement sur le bénévolat, avec un public adulte et une aisance numérique très hétérogène.
    \item \textbf{Où :} Application web mobile-first, accessible également sur ordinateur et tablette, déployée d’abord dans un club pilote (Aquaclub21).
    \item \textbf{Quand :} Projet développé du 25 septembre 2025 au 6 juin 2026 (POC $\rightarrow$ MVP $\rightarrow$ Bêta $\rightarrow$ V1).
    \item \textbf{Comment :} Architecture 3-tiers \textit{(Next.js - Nest.js - PostgreSQL + MongoDB)} avec intégration HelloAsso en V1 et pilotage via GitHub Projects.
    \item \textbf{Pourquoi :} Remplacer la fragmentation des outils actuels par une solution unique, simple et accessible, afin de réduire la charge administrative et d’améliorer l’expérience des adhérents.
\end{itemize}

\vspace{0.5cm}

\textbf{Pitch QQOQCP :} 
Je développe une application web mobile-first destinée aux clubs associatifs de plongée pour centraliser la gestion des adhérents, des certificats médicaux, des événements, des inscriptions, des rôles et des paiements.

L’application repose sur une architecture 3-tiers \textit{(Next.js - Nest.js - PostgreSQL + MongoDB)} et sera testée en conditions réelles dans un club pilote (Aquaclub21) avant une version finale début juin 2026.

L’objectif est de remplacer la multiplicité des outils actuels par une solution unique, fiable et simple d’usage, afin de réduire la charge administrative des bénévoles et d’améliorer l’expérience des adhérents lors de l'utilisation de l'application.
\vspace{0.5cm}
\section{Problématique et objectifs SMART}

\noindent\textbf{Problématique centrale :}\\[0.2cm]
\textit{Comment concevoir et déployer une plateforme web simple, centralisée et sécurisée permettant de réduire d’au moins 50~\% la charge administrative des clubs de plongée affiliés à la FFESSM, tout en restant accessible à un public majoritairement peu technophile et hétérogène en termes d’usages numériques~?
}

Cette problématique s’illustre notamment à travers le profil d’adhérents peu à l’aise avec les outils numériques, comme Marc, 52 ans, plongeur loisir, qui rencontre des difficultés à gérer des démarches administratives réparties sur plusieurs supports. La fragmentation des outils et le manque d’ergonomie constituent pour ce type d’utilisateur un frein à l’inscription et à la participation aux activités du club. Ce constat a guidé les choix fonctionnels et ergonomiques du projet.


\vspace{0.5cm}
La gestion quotidienne d’un club associatif de plongée repose aujourd’hui sur une multitude d’outils hétérogènes : VPDive, tableurs, formulaires externes, Google Drive, messageries, documents papier.
Cette fragmentation génère :
\begin{itemize}
    \item une navigation complexe pour un public majoritairement adulte et peu technophile.
    \item des manipulations manuelles répétitives,
    \item une perte ou duplication d’informations,
    \item un risque d’erreur élevé,
    \item une surcharge administrative estimée entre 1 et 3 heures par semaine pour les bénévoles du comité,
\end{itemize}

Au sein du club pilote (Aquaclub21), 64{,}6\,\% des adhérents ont plus de 46 ans dont 27{,}1\,\% plus de 61 ans, ce qui renforce le besoin d’une interface simple, lisible et guidée.

Ces difficultés impactent :
\begin{itemize}
    \item l’expérience utilisateur,
    \item la satisfaction globale des adhérents,
    \item la qualité du suivi administratif (certificats, paiements, inscriptions),
    \item la sécurité (certificats non valides),
    \item la participation aux activités (mauvaises informations, pertes d’inscriptions),
\end{itemize}

\vspace{0.5cm}
En tant que membre du comité, je suis confronté directement à ces problèmes, ce qui m’a permis d’identifier avec précision les points de friction, leurs impacts et les besoins concrets du terrain.


\vspace{0.5cm}

\subsection*{Objectifs SMART}

\noindent{Les objectifs du projet ont été formalisés selon la méthode SMART
(Spécifique, Mesurable, Atteignable, Réaliste et Temporellement défini)
afin de garantir un pilotage précis et vérifiable.}

\vspace{0.3cm}

\subsection*{Indicateurs de succès quantifiés (KPIs)}

Afin d’évaluer objectivement la réussite globale du projet, trois indicateurs clés (KPIs) ont été définis, chacun associé à une valeur cible et à un mode de mesure :

\begin{itemize}
    \item \textbf{Taux d’adoption} : au moins \textbf{20 utilisateurs actifs} durant la phase bêta.
    \textit{Mesure : analyse des logs de connexion et du nombre de comptes actifs.}

    \item \textbf{Gain de temps et réduction des relances} : diminuer d’au moins \textbf{50\%} le volume de relances manuelles sous 2 mois.
    \textit{Mesure : comptage hebdomadaire des relances manuelles réalisées par le comité.}

    \item \textbf{Satisfaction utilisateur} : atteindre \textbf{90\% de satisfaction} en fin de phase bêta.
    \textit{Mesure : questionnaire anonyme envoyé aux adhérents du club pilote.}
\end{itemize}

\vspace{0.5cm}


\noindent\textbf{Objectif SMART 1 — Accessibilité des adhérents non technophiles}

\noindent\textbf{Indicateur de réussite :} 80\% des adhérents complètent l’inscription sans assistance, mesuré via le taux de complétion et le nombre de demandes d’aide (objectif $<$ 10 demandes).

\begin{itemize}
  \item \textbf{Spécifique :} Simplifier l’usage de la plateforme pour les adhérents peu à l’aise avec le numérique.
  \item \textbf{Mesurable :} Atteindre \textbf{80\% de parcours complets} réalisés sans assistance, mesurés via le taux de complétion et le nombre de demandes d’aide.
  \item \textbf{Atteignable :} Parcours guidé, interface mobile-first, formulaires simplifiés, notifications claires.
  \item \textbf{Réaliste :} Besoin confirmé par l’analyse utilisateurs du club (moyenne d’âge 48+ ans).
  \item \textbf{Temporel :} Mesuré pendant la phase bêta (\textbf{février $\rightarrow$ avril 2026}).
\end{itemize}

\vspace{0.5cm}

\noindent\textbf{Objectif SMART 2 — Réduction de la charge administrative}

\noindent\textbf{Indicateur de réussite :} Temps de gestion réduit à moins de 60 minutes/semaine, mesuré via un suivi hebdomadaire du temps passé par les bénévoles.

\begin{itemize}
  \item \textbf{Spécifique :} Réduire le temps consacré aux relances et au suivi manuel (certificats, inscriptions).
  \item \textbf{Mesurable :} Réduire le temps administratif de \textbf{1--3 h/semaine à 30--60 min/semaine}, mesuré via le suivi du temps passé et les relances automatiques vs manuelles.
  \item \textbf{Atteignable :} Automatisation des relances, centralisation des données, flux simplifiés.
  \item \textbf{Réaliste :} Charge actuelle estimée à 1--3 h/semaine par bénévole.
  \item \textbf{Temporel :} Mesure effectuée entre \textbf{février $\rightarrow$ avril 2026}.
\end{itemize}

\vspace{0.5cm}

\noindent\textbf{Objectif SMART 3 — Adoption de la Bêta par le club pilote}

\noindent\textbf{Indicateur de réussite :} Au moins 20 utilisateurs actifs et 20 inscriptions à des événements dans le mois suivant le déploiement de la bêta.

\begin{itemize}
  \item \textbf{Spécifique :} Déployer une version fonctionnelle couvrant profil, certificats, événements et inscriptions.
  \item \textbf{Mesurable :} Atteindre \textbf{$\geq$ 20 utilisateurs actifs} et \textbf{$\geq$ 20 inscriptions}, mesurés via logs de connexion et statistiques.
  \item \textbf{Atteignable :} Périmètre MVP limité et validé.
  \item \textbf{Réaliste :} Technologies maîtrisées (Next.js / Nest.js / PostgreSQL).
  \item \textbf{Temporel :} Objectif mesuré entre \textbf{février $\rightarrow$ avril 2026}.
\end{itemize}

\vspace{0.5cm}

\noindent\textbf{Objectif SMART 4 — Performance et stabilité de l’API}

\noindent\textbf{Indicateur de réussite :} 95\% des requêtes serveur sous 300\,ms, mesuré via l’outil de monitoring en production.

\begin{itemize}
  \item \textbf{Spécifique :} Garantir une API performante et stable pour une expérience fluide.
  \item \textbf{Mesurable :} Assurer que \textbf{95\% des requêtes} répondent en moins de \textbf{300\,ms}.
  \item \textbf{Atteignable :} Optimisation des endpoints, mise en cache, bonnes pratiques Nest.js.
  \item \textbf{Réaliste :} Aligné avec les standards des architectures 3-tiers.
  \item \textbf{Temporel :} Mesuré en continu à partir de la \textbf{Bêta (février 2026)}.
\end{itemize}

\vspace{0.5cm}

\noindent\textbf{Objectif SMART 5 — Sécurisation des données et conformité RGPD}

\noindent\textbf{Indicateur de réussite :} 100\% des routes sensibles protégées et aucun incident de sécurité durant la phase bêta.

\begin{itemize}
  \item \textbf{Spécifique :} Protéger les données personnelles des adhérents.
  \item \textbf{Mesurable :} Vérifier que \textbf{100\% des routes sensibles} sont protégées (JWT + RBAC), via audit interne.
  \item \textbf{Atteignable :} Cookies HttpOnly, RBAC, validation stricte des schémas.
  \item \textbf{Réaliste :} Aligné avec les exigences légales du RGPD.
  \item \textbf{Temporel :} Implémenté pour la \textbf{Bêta Release (avril 2026)}.
\end{itemize}


\subsection{Synthèse des objectifs SMART et indicateurs de performance}

\begin{center}
\begin{tabular}{|p{4.5cm}|p{4.2cm}|p{3cm}|p{3.3cm}|}
\hline
\textbf{Objectif SMART} & \textbf{KPI associé} & \textbf{Mesure cible} & \textbf{Temporalité} \\
\hline
Accessibilité des adhérents non technophiles &
Taux de complétion sans assistance &
80~\% des parcours complets, $<$ 10 demandes d’aide &
Phase bêta (février $\rightarrow$ avril 2026) \\
\hline
Réduction de la charge administrative &
Temps de gestion administrative hebdomadaire &
30--60 min/semaine (réduction $\geq$ 50~\%) &
Février $\rightarrow$ avril 2026 \\
\hline
Adoption de la bêta par le club pilote &
Nombre d’utilisateurs actifs et inscriptions &
$\geq$ 20 utilisateurs actifs et $\geq$ 20 inscriptions &
1 mois après déploiement de la bêta \\
\hline
Performance et stabilité de l’API &
Latence des requêtes serveur (P95) &
95~\% des requêtes $<$ 300\,ms &
Mesure continue à partir de février 2026 \\
\hline
Sécurisation des données et conformité RGPD &
Couverture des routes sensibles et incidents &
100~\% des routes protégées, 0 incident &
Bêta Release (avril 2026) \\
\hline
\end{tabular}
\end{center}


\vspace{0.5cm}

\textbf{Bénéfices attendus :}
L’application a pour objectif de centraliser l’ensemble des opérations nécessaires à la gestion d’un club de plongée dans un seul outil, afin de simplifier le travail des encadrants et d’améliorer l’expérience des adhérents. Elle remplace plusieurs outils dispersés par une plateforme unique, intuitive et maintenable dans le temps. La gestion des adhérents, des événements, des certificats et des inscriptions est unifiée, ce qui réduit les risques d’erreur et de perte d’information. Grâce à une interface adaptée aux publics peu familiers du numérique, l’application fluidifie la participation aux activités et diminue les frictions administratives. Enfin, la solution pourra évoluer à l’échelle fédérale afin d’assurer une cohérence numérique au sein des structures FFESSM.

\vspace{0.5cm}

\textbf{Diagramme de contexte :}

\vspace{0.5cm}


\begin{figure}[H]
    \centering
    \includegraphics[height=14cm, keepaspectratio]{assets/diagramme_contexte_doc.png}
    \caption{Diagramme de contexte du projet}
\end{figure}

\paragraph{Analyse du diagramme de contexte}

Le diagramme de contexte met en évidence les trois acteurs principaux du système : les adhérents, les encadrants et le comité directeur du club. Il illustre les flux essentiels liés aux processus administratifs du club, notamment les inscriptions, la gestion des certificats médicaux, la participation aux événements et les paiements. Cette représentation permet d’identifier clairement les responsabilités de chaque acteur ainsi que les échanges critiques à centraliser. Elle justifie la mise en place d’une plateforme unique afin de réduire la fragmentation des outils existants, de fiabiliser les données et de limiter les interventions manuelles, en cohérence avec la problématique et les objectifs métier du projet.


\section{Liens utiles}

\begin{itemize}
    \item GitHub About: \url{https://docs.github.com/}
    \item SMART Goals: \url{https://bit.ly/smart-goals-atlassian}
    \item Project Management Institute: \url{https://www.pmi.org/}
    \item Agile Manifesto: \url{https://agilemanifesto.org/}
    \item Business Model Canvas: \url{https://bit.ly/business-model-canvas}
    \item FFESSM — Bilan à 700 jours (paragraphe "Amélioration de notre système informatique")
\url{https://ffessm.fr/actualites/bilan-de-la-ffessm-a-700-jours}
\item Nombre de clubs de plongée en France: \url{https://ffessm.fr/actualites/un-ete-immersif-pour-les-francais}
    \item Données sur l'âge des plongeurs en 2002 page 18: \url{https://ffessm.fr/uploads/media/docs/0001/02/9b74472404c10cd0bd3c09bf15f1a9b217159689.pdf}
    \item Données sur l'âge des plongeurs en 2020 page 11: \url{https://www.dirm.mediterranee.developpement-durable.gouv.fr/IMG/pdf/mnca_2019_etudeplongee_version_finale.pdf}
  \end{itemize}

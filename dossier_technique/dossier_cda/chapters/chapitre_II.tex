\chapter{Cadrage et cahier des charges}

% =========================
% 2.1 Objectifs métier, techniques et pédagogiques
% =========================
\section{Objectifs métier, techniques et pédagogiques}


Cette section présente les objectifs principaux du projet \textit{Diving O Club} selon trois axes complémentaires : métier, technique et pédagogique. Ces objectifs guident l’ensemble du cadrage du projet et structurent les livrables jusqu’à la version finale.

\subsection*{Objectifs métier}

Les objectifs métier répondent aux besoins concrets d’un club associatif de plongée fonctionnant avec des ressources bénévoles :

\begin{itemize}
    \item \textbf{Centraliser l’ensemble de la gestion du club} : adhérents, événements, certificats, inscriptions et paiements, aujourd’hui dispersés dans plusieurs outils (Excel, documents papier, e-mails).
    \item \textbf{Réduire d’au moins 50\% la charge administrative} liée aux inscriptions, relances, vérifications de documents et suivi des obligations fédérales.
    \item \textbf{Garantir une expérience utilisateur simple et accessible}, adaptée y compris aux adhérents peu familiers du numérique.
\end{itemize}

Ces objectifs sont mesurables à travers :
\begin{itemize}
    \item un \textbf{taux d’adoption cible de 20 utilisateurs actifs} durant la phase bêta ;
    \item une \textbf{diminution de 50\% des relances manuelles} dans les deux premiers mois ;
    \item un \textbf{niveau de satisfaction de 90\%} sur le questionnaire de fin de bêta.
\end{itemize}

\subsection*{Objectifs techniques}

Les objectifs techniques définissent le cadre technologique et les exigences de qualité logicielle :

\begin{itemize}
    \item \textbf{Concevoir une architecture trois-tiers scalable} : Frontend (Next.js / React), API backend (Node.js / Nest.js) et base de données (PostgreSQL complété par MongoDB pour certains usages analytiques).
    \item \textbf{Mettre en place une authentification sécurisée} reposant sur JWT, cookies HttpOnly et un contrôle des accès par rôle (RBAC), en conformité avec les bonnes pratiques et le RGPD.
    \item \textbf{Assurer des performances maîtrisées} : disponibilité cible de 99\,\% en bêta et temps de réponse moyen inférieur à 300\,ms pour 95\,\% des requêtes.
    \item \textbf{Intégrer des mécanismes de robustesse} : sauvegardes régulières, monitoring (logs MongoDB), journalisation des actions sensibles et préparation à l’isolation multi-tenant pour une future évolution multi-clubs.
\end{itemize}

\subsection*{Objectifs pédagogiques (CDA)}

Le projet s’inscrit dans le cadre du titre \textit{Concepteur Développeur d’Applications}. Il constitue une mise en situation professionnelle complète permettant de mobiliser l’ensemble des compétences visées :

\begin{itemize}
    \item \textbf{Mettre en œuvre l’ensemble des compétences du titre CDA} dans un contexte réel : analyse, conception, développement, sécurisation, documentation.
    \item \textbf{Concevoir et documenter une architecture complète} : modèles de données (MCD/MLD), user stories, diagrammes, backlog, documentation API, CI/CD.
    \item \textbf{Développer et déployer une application sécurisée} incluant une gestion avancée des rôles et du traitement des données personnelles.
    \item \textbf{Interagir avec un client associatif réel} (club pilote), recueillir ses besoins, intégrer ses retours et adapter le développement en continu.
\end{itemize}

Ces objectifs forment un socle cohérent aligné avec les contraintes du projet (ressource unique, disponibilité limitée du comité, environnement associatif bénévole) et garantissent une base solide pour l’évolution future du produit.

\vspace{0.5cm}

% =========================
% 2.2 Cibles et parties prenantes
% =========================
\section{Cibles et parties prenantes}

Cette section identifie les différents types d’utilisateurs finaux de l’application ainsi que les
parties prenantes impliquées dans le projet. Elle constitue une base essentielle pour comprendre
les besoins, les contraintes et l’influence de chaque acteur sur l’évolution de la solution.

\subsection*{Types d’utilisateurs}

Les utilisateurs du système se répartissent en quatre grandes catégories. Cette classification
présente les rôles et usages principaux sans entrer dans le détail, lequel est développé dans
les personae qui suivent.

\begin{enumerate}
  \item \textbf{Nouveau utilisateur / sans club} \\
  Personne souhaitant découvrir la plongée ou rejoindre un club FFESSM. Premiers usages :
  création de compte, recherche de club, demande d’inscription.

  \vspace{0.2cm}

  \item \textbf{Adhérent (pratiquant plongée / apnée)} \\
  Licencié FFESSM, utilise l’application pour s’inscrire aux activités et les payer, gérer son certificat médical.

  \vspace{0.2cm}

  \item \textbf{Moniteur / encadrant} \\
  Encadrant FFESSM, responsable de l’organisation, du contrôle des certificats et du
  déroulement des activités.

  \vspace{0.2cm}

  \item \textbf{Membre du comité directeur} \\
  Président, trésorier, secrétaire ou responsable administratif. Supervise les adhésions,
  paiements, validations et données du club.
\end{enumerate}

\vspace{0.4cm}

\subsection*{Personae détaillés}

Les personae permettent de représenter plus finement des profils représentatifs des utilisateurs
réels, en intégrant leurs motivations, irritants, comportements et besoins spécifiques. Deux
personae ont été définis pour guider les choix de conception.

\vspace{0.5cm}

% ---------------------------------------------------------------
% PERSONA 1
% ---------------------------------------------------------------

\subsubsection*{Persona 1 — Marc, 52 ans, Adhérent régulier (plongée bouteille)}

\textbf{Profil :}  
Marc est adhérent depuis 7 ans dans un club FFESSM. Il travaille dans le bâtiment et utilise 
peu les outils numériques. Il consulte son téléphone surtout pour ses mails et WhatsApp.  
Il participe régulièrement aux sorties techniques et aux entraînements piscine.

\textbf{Motivations :}
\begin{itemize}
  \item S’inscrire facilement aux entraînements et sorties.
  \item Avoir un suivi clair de son certificat médical.
  \item Payer facilement depuis son téléphone les sorties.
\end{itemize}

\textbf{Frustrations / irritants :}
\begin{itemize}
  \item Difficulté à retrouver les informations éparpillées (Drive, mails, PDF).
  \item Manque d'indications sur les étapes à suivre pour certaines démarches.
  \item Perte de temps lorsqu’il n’arrive pas à s’inscrire ou que son certificat n’est pas à jour.
\end{itemize}

\textbf{Besoins numériques :}
\begin{itemize}
  \item Interface simple, lisible, mobile-first.
  \item Parcours guidé pour l’inscription aux événements.
  \item Notifications claires en cas de certificat expiré ou dossier incomplet.
\end{itemize}

\textbf{Objectifs personnels :}
\begin{itemize}
  \item Pratiquer la plongée sans stress administratif.
  \item Pouvoir s’inscrire rapidement et ne rien oublier.
  \item Être autonome, même sans grande maîtrise informatique.
\end{itemize}

\vspace{0.5cm}

% ---------------------------------------------------------------
% PERSONA 2
% ---------------------------------------------------------------

\subsubsection*{Persona 2 — Sophie, 38 ans, Encadrante (E2) et membre du comité}

\textbf{Profil :}  
Sophie est monitrice fédérale (E2) et membre du comité depuis 3 ans. Elle organise chaque mois 
des sorties techniques, gère les validations de certificats et les listes d’inscrits.  
Elle utilise régulièrement son ordinateur et son smartphone, et recherche de l’efficacité.

\textbf{Motivations :}
\begin{itemize}
  \item Gagner du temps dans la gestion administrative du club.
  \item Suivre en un coup d’œil les certificats, paiements et inscriptions.
  \item Avoir un outil fiable pour éviter les erreurs critiques lors des sorties.
\end{itemize}

\textbf{Frustrations / irritants :}
\begin{itemize}
  \item Qualité et fiabilité inégales des documents envoyés (PDF, photos).
  \item Dossiers incomplets, certificats expirés, relances manuelles répétitives.
  \item Outils dispersés rendant difficile la coordination entre encadrants.
\end{itemize}

\textbf{Besoins numériques :}
\begin{itemize}
  \item Tableau de bord clair pour suivre les adhérents et leurs statuts.
  \item Centralisation des certificats, inscriptions et paiements.
  \item Filtrage, notifications automatiques et système anti-erreur.
\end{itemize}

\textbf{Objectifs professionnels :}
\begin{itemize}
  \item Assurer la sécurité des pratiquants.
  \item Réduire le temps administratif pour se concentrer sur l’encadrement.
  \item Avoir un outil fiable pour la coordination avec le comité.
\end{itemize}

\vspace{0.6cm}

% ---------------------------------------------------------------
% PARTIES PRENANTES
% ---------------------------------------------------------------

\subsection*{Parties prenantes : analyse influence / intérêt}

La réussite du projet Diving O Club dépend de différentes parties prenantes ayant des niveaux
d’influence et d’intérêt variés. Leur identification permet d’adapter la stratégie d’implication
et de communication tout au long du projet.

\subsubsection*{Liste des parties prenantes}

\begin{itemize}
  \item \textbf{Adhérents} : utilisateurs finaux, concernés par la simplicité d’usage et la clarté des parcours.
  \item \textbf{Moniteurs / encadrants} : acteurs opérationnels dont l’usage quotidien conditionne l’adoption interne.
  \item \textbf{Comité directeur} : décideurs métier, garants de la conformité et des priorités internes.
  \item \textbf{Développeur} : responsable technique du projet, conception, qualité, sécurité et maintenance.
  \item \textbf{HelloAsso} : service tiers pour les paiements en ligne, intégré au fonctionnement administratif.
\end{itemize}


\subsubsection*{Analyse influence / intérêt}
Cette analyse permet d’identifier les acteurs stratégiques et de hiérarchiser les besoins :

\begin{itemize}
  \item \textbf{Forte influence, fort intérêt} : comité directeur, développeur.  
  Ils orientent les priorités, arbitrent le MVP et conditionnent la qualité finale.

  \item \textbf{Influence moyenne, intérêt élevé} : moniteurs / encadrants.  
  Ils influencent directement l’usage sur le terrain et l’adéquation aux pratiques du club.

  \item \textbf{Faible influence, intérêt moyen / élevé} : adhérents.  
  Leur satisfaction détermine l’adoption à long terme, en particulier sur le mobile-first.

  \item \textbf{Influence faible, intérêt moyen} : HelloAsso.  
  Fournit un service critique (paiements), mais sans impact direct sur la stratégie.
\end{itemize}

\vspace{0.4cm}
\subsubsection*{Impact sur le périmètre du MVP}

L’analyse des parties prenantes permet également de définir avec précision les fonctionnalités qui doivent impérativement figurer dans le MVP. Les besoins critiques du comité directeur (certificats, paiements, inscriptions) et les irritants majeurs identifiés chez les adhérents (parcours complexe, manque de centralisation) ont orienté les priorités suivantes :

\begin{itemize}
    \item parcours d'inscription simplifié pour les adhérents ;
    \item centralisation certificats–paiements–inscriptions pour le comité ;
    \item visibilité immédiate des inscrits et des documents pour les encadrants.
\end{itemize}

Ainsi, le MVP se concentre uniquement sur les modules apportant une valeur directe aux parties prenantes clés : authentification, gestion des activités, inscriptions, certificats médicaux et synchronisation des paiements.


\subsubsection*{Tableau d'analyse des parties prenantes}
\begin{table}[H]
\centering
\small
\begin{tabular}{|p{3.5cm}|p{4.5cm}|p{3.5cm}|p{5cm}|}
\hline
\textbf{Partie prenante} & \textbf{Intérêt pour le projet} & \textbf{Influence} & \textbf{Commentaire} \\ \hline

Comité directeur & Très élevé & Très élevé & Définit les priorités métier, valide les choix stratégiques. \\ \hline

Moniteurs / encadrants & Élevé & Moyen & Leur usage quotidien conditionne l’adoption opérationnelle. \\ \hline

Adhérents & Moyen & Faible & Leur expérience utilisateur détermine la satisfaction à long terme. \\ \hline

Développeur (Kevin) & Très élevé & Élevé & Garantit qualité technique, sécurité et livraison dans les délais. \\ \hline

HelloAsso & Moyen & Faible & Service indispensable au paiement, doit être surveillé. \\ \hline
\end{tabular}
\end{table}

\subsubsection*{Matrice pouvoir / intérêt et stratégies associées}

La matrice pouvoir / intérêt permet de définir une stratégie d’implication adaptée à chacune
des parties prenantes.

\begin{table}[H]
\centering
\small
\begin{tabular}{|p{4cm}|p{3cm}|p{3cm}|p{6cm}|}
\hline
\textbf{Partie prenante} & \textbf{Pouvoir} & \textbf{Intérêt} & \textbf{Stratégie de gestion} \\ \hline

Comité directeur & Élevé & Élevé & Impliquer de près : revues bimensuelles, arbitrage des priorités, validation des étapes. \\ \hline

Moniteurs / encadrants & Moyen & Élevé & Consulter régulièrement : tests mensuels, retours sur l’ergonomie et les parcours. \\ \hline

Adhérents & Faible & Moyen & Informer : parcours guidé, interface simple, tutoriels, feedback utilisateur en continu. \\ \hline

HelloAsso & Faible & Moyen & Surveiller : tests sandbox, plans de reprise en cas d’incident, monitoring des paiements. \\ \hline

\end{tabular}
\end{table}

\vspace{0.4cm}
\subsubsection*{Risques liés aux parties prenantes}

L'analyse met également en lumière plusieurs risques qui peuvent impacter l’adoption du produit :

\begin{itemize}
    \item \textbf{Adhérents peu technophiles} : risque de non-adoption si l’interface n’est pas suffisamment simple et mobile-first.
    \item \textbf{Encadrants surchargés} : faible disponibilité pour tester ou remonter des retours, ce qui peut ralentir l’itération.
    \item \textbf{Comité directeur} : risque de surcharge administrative si la synchronisation paiements–inscriptions n’est pas fiable.
    \item \textbf{HelloAsso} : dépendance externe ; un changement d’API ou une indisponibilité peut bloquer certaines opérations.
\end{itemize}

La roadmap et le périmètre du MVP intègrent ces risques afin de garantir une adoption progressive, sécurisée et réaliste au sein du club pilote.


\subsubsection*{Conclusion}
Cette analyse garantit une communication adaptée et une prise en compte fine des besoins
des différents acteurs. Elle permet d’ajuster le périmètre du MVP, de construire une roadmap
cohérente et d’assurer une adoption optimale par les utilisateurs stratégiques du club.



\clearpage

% =========================
% 2.3 Exigences fonctionnelles
% =========================
\section{Exigences fonctionnelles}

\subsection{Fonctionnalités « Front Office »}
\textbf{Front Office :} 
\begin{itemize}
  \item Création de compte / authentification
  \item Création d’un club : un utilisateur peut proposer un nouveau club, qui sera placé en « attente de validation ».
  \item Suivi du statut de validation du club (en attente / validé / refusé).
  \item Gestion du profil utilisateur (coordonnées, licence, certificat médical)
  \item Calendrier des événements (consultation et inscription / modification / annulation)
  \item Paiement en ligne via HelloAsso pour les événements payants
  \item Campagne d’adhésion (cotisation) : formulaire HelloAsso, ajout au \textbf{panier}, paiement et attestation
  \item Panier (événements \textit{et} items boutique/adhésion) : visualisation, suppression, paiement
  \item Panier pour consulter la liste des activités qu’il reste à payer
  \item Consultation de l’historique personnel (inscriptions, paiements, historique, certificats)
  \item Accès à la page du club
\end{itemize}

\subsection{Fonctionnalités « Back Office »}
\textbf{Back Office :}
\begin{itemize}
  \item Gestion des utilisateurs (invitation, modification de profil, désactivation)
  \item Gestion des rôles (adhérent / moniteur / comité directeur / administrateur du club)
  \item Gestion des clubs :
  \begin{itemize}
    \item validation des clubs nouvellement créés (admin technique)
    \item attribution automatique du rôle « administrateur du club » au créateur
    \item modification des informations du club
  \end{itemize}
  \item Gestion des événements (création, modification, suppression, capacité, tarifs, règles)
  \item Validation des certificats médicaux
  \item Suivi des inscriptions (confirmées / annulées)
  \item Suivi des paiements (via HelloAsso + marquage manuel si besoin pour règlement par espèces ou chèque)
  \item Configuration du club (informations, page publique, visuels)
\end{itemize}

\subsection*{Spécification des fonctionnalités par couche}

Cette section détaille les fonctionnalités du MVP en les répartissant entre les trois couches
de l’architecture : \textbf{front-end}, \textbf{back-end} et \textbf{API REST}.  
Cette structuration permet de clarifier les responsabilités de chaque couche et de garantir une
cohérence entre interface, logique métier et points d’accès.

\begin{table}[H]
\centering
\small
\setlength{\tabcolsep}{3pt}
\renewcommand{\arraystretch}{1.2}
\begin{tabular}{|p{1.8cm}|p{3.6cm}|p{6.2cm}|p{3.9cm}|}
\hline
\textbf{Couche} & \textbf{Fonctionnalité} & \textbf{Détails / critères} & \textbf{API} \\
\hline

Front & Authentification & Formulaire email + mot de passe, messages d'erreur, redirection tableau de bord & \texttt{POST /auth/login} \\
\hline

Front & Inscription & Formulaire nom, email, mot de passe (avec confirmation) ; validations côté client & \texttt{POST /auth/register} \\
\hline

Front & Création de club & Formulaire nom + email ; configuration initiale ; validation client & \texttt{POST /clubs} \\
\hline

Front & Recherche de club & Champ de recherche, liste filtrée, suggestions, affichage rapide & \texttt{GET /clubs?search=} \\
\hline

Front & Page d’un club & Affichage nom, description, encadrants, activités ; bouton “Rejoindre le club” & \texttt{GET /clubs/:id} \\
\hline

Back & Gestion clubs & Création tenant, assignation rôle owner, validations métier, récupération des infos club & \texttt{GET /clubs/:id} \\
\hline

API & Liste des clubs & Renvoie les clubs, avec filtre, tri et pagination & \texttt{GET /clubs} \\
\hline

Back & Gestion des rôles & Attribution des rôles, contrôle d’accès aux routes, journaux d’accès & (interne) \\
\hline

Back & Création d’événement & CRUD événement ; affichage dans le calendrier ; contrôles métier & \texttt{GET /events}, \texttt{POST /events}, \texttt{PATCH /events/:id} \\
\hline

Front & Inscription à un événement & Bouton actif si certificat valide et places disponibles ; messages d’état & \texttt{POST /registrations} \\
\hline

API & Paiements HelloAsso & Réception webhook ; état payé/non payé ; id transaction stocké & \texttt{POST /webhooks/helloasso} \\
\hline

Back & Certificat médical & Upload ; statuts en attente / validé / refusé ; historique des vérifications & \texttt{POST /certificates}, \texttt{PATCH /certificates/:id/status} \\
\hline

Front & Notifications & Bannières d’alerte et emails transactionnels ; préférences utilisateur & \texttt{POST /notifications/test} \\
\hline

Back & Administration minimale & Liste inscrits, validations, filtres, export simplifié & \texttt{GET /admin/overview} \\
\hline

\end{tabular}
\normalsize
\end{table}



\subsection*{User stories (format normé)}

Les besoins fonctionnels du projet sont exprimés sous forme de \textbf{User Stories},
selon la structure standardisée issue des pratiques Agile :

\begin{center}
\textit{En tant que [rôle], je veux [objectif] afin de [bénéfice].}
\end{center}

Ce format unifié garantit une compréhension partagée entre les profils métier et
techniques. Toutes les User Stories du projet sont rédigées selon cette norme et
organisées dans GitHub Project (backlog complet : EPIC $ \rightarrow $ US $ \rightarrow $ tâches).

\vspace{0.4cm}
\subsubsection*{Exemple de User Story détaillée (US-00)}

\textbf{US-00 — Création de compte utilisateur}

\textbf{User Story} \\
En tant qu'\textbf{utilisateur externe non inscrit}, je veux créer un compte via un
formulaire simple (nom, email, mot de passe) afin d'accéder à mon espace personnel
et rejoindre ou créer un club.

\vspace{0.3cm}
\textbf{Critères d'acceptation (format Gherkin)}

\begin{verbatim}
Scenario: Inscription réussie
  Given un utilisateur non inscrit
  When il saisit un nom, un email valide et un mot de passe sécurisé
  Then son compte est créé
  And il est automatiquement authentifié
  And il est redirigé vers son tableau de bord

Scenario: Email déjà utilisé
  Given un utilisateur non inscrit
  When il saisit un email déjà existant
  Then un message "Email déjà utilisé" apparaît
  And la création est refusée

Scenario: Mot de passe invalide
  Given un utilisateur non inscrit
  When il saisit un mot de passe ne respectant pas la politique
  Then un message d’erreur apparaît
\end{verbatim}

\vspace{0.2cm}
\textbf{Backend :}
\begin{itemize}
  \item Endpoint \texttt{POST /auth/register}
  \item Validation DTO (email unique, règles mot de passe)
  \item Hash bcrypt
  \item Création utilisateur en base
  \item Génération des tokens (access + refresh)
  \item Réponse JSON + refresh token en cookie \texttt{httpOnly}
\end{itemize}

\textbf{Frontend :}
\begin{itemize}
  \item Formulaire (nom, email, mot de passe, confirmation)
  \item Validation en direct
  \item Affichage des erreurs
  \item Redirection tableau de bord
\end{itemize}

\vspace{0.4cm}
\subsubsection*{User Stories du MVP}

L'ensemble des User Stories (US-01 à US-08) suivantes sont rédigées sur le même
format normé et détaillées dans le \textbf{GitHub Project du MVP}. Elles couvrent :

\begin{itemize}
  \item \textbf{US-01} : Connexion
  \item \textbf{US-02} : Création d'événement
  \item \textbf{US-03} : Inscription à une activité
  \item \textbf{US-04} : Gestion du certificat médical
  \item \textbf{US-05} : Paiement via HelloAsso
  \item \textbf{US-06} : Suivi administratif minimal
  \item \textbf{US-07} : Création de club
  \item \textbf{US-08} : Validation d’un club par l’admin technique
\end{itemize}

Chacune comporte :
\begin{itemize}
  \item une formulation standardisée “En tant que… je veux… afin de…”,
  \item des critères d'acceptation Gherkin,
  \item et un découpage technique associé (tâches Backend/Frontend).
\end{itemize}

\subsection*{Mesure d’impact par fonctionnalité}

Chaque fonctionnalité du MVP est associée à un indicateur mesurable (KPI) permettant
d’évaluer son impact réel sur l’organisation du club pilote. Ces indicateurs complètent
les objectifs SMART et permettent un suivi opérationnel concret.

\begin{table}[H]
\centering
\small
\setlength{\tabcolsep}{4pt}
\renewcommand{\arraystretch}{1.2}
\begin{tabular}{|p{4cm}|p{6cm}|p{6cm}|}
\hline
\textbf{Fonctionnalité} & \textbf{KPI associé} & \textbf{Méthode de mesure} \\ \hline

Authentification / création de compte &
Taux de complétion du parcours & 
Logs de création de compte ; \% d'utilisateurs allant jusqu’à la fin du formulaire. \\ \hline

Création / gestion des clubs &
Nombre de clubs créés dans le mois suivant la Beta &
Analyse des enregistrements \texttt{Club} et des rôles associés. \\ \hline

Inscriptions aux événements &
Taux d’inscriptions enregistrées via la plateforme (>80\%) &
Comparaison inscriptions plateforme vs listes WhatsApp/Excel. \\ \hline

Certificats médicaux &
Réduction des certificats expirés non détectés (objectif : -70\%) &
Logs de validation, indicateurs avant/après déploiement. \\ \hline

Synchronisation HelloAsso &
Taux d’erreurs dans la correspondance paiement/inscription (objectif : <5\%) &
Analyse journalière des statuts paiement vs inscription. \\ \hline

Calendrier / événements &
Nombre de connexions au calendrier et inscriptions issues du planning &
Statistiques d’affichage \texttt{/events} et des clics sur “S’inscrire”. \\ \hline

Administration comité &
Réduction du temps administratif hebdomadaire (objectif : -50\%) &
Auto-déclaration + mesure du temps passé avant/après via journal hebdo. \\ \hline
\end{tabular}
\normalsize
\end{table}

Ces indicateurs permettent d’évaluer objectivement l’impact du MVP sur le fonctionnement
du club pilote et d’orienter les priorités pour la version Beta et la V1.


\subsection{Découpage Front / Back (fonctionnalités clés)}
\begin{table}[h!]
\centering
\begin{tabular}{|p{5.5cm}|p{5.5cm}|p{5.5cm}|}
\hline
\textbf{Fonction} & \textbf{Front (Next.js/React)} & \textbf{Back (Nest.js)} \\ \hline
Auth / Sessions & Formulaires, états UI, garde routes & Endpoints /login, /refresh, /logout ; JWT, cookies httpOnly \\ \hline
Événements & Liste, détail, formulaires CRUD & Endpoints /events (CRUD), règles capacité \\ \hline
Inscriptions & Bouton s'inscrire/annuler, vues perso & /registrations, contrôle certificat/capacité \\ \hline
Certificats & Upload, aperçu statut & /certificates (upload, statut), validations comité \\ \hline
Paiements & Redirection, retour statut & Webhooks/retour HelloAsso, persistance statut \\ \hline
Admin minimal & UI listes (users, events, validations) & Endpoints admin sécurisés + logs \\ \hline
\end{tabular}
\end{table}


\subsection{L'utilisateur (public)}
\textbf{Les types d'utilisateurs :}
\textbf{Rôles et permissions utilisateurs :}\\[0.2cm]

\begin{itemize}
  \item \textbf{Utilisateurs sans club (externes)} : peuvent compléter leur profil, créer un club, rechercher et voir les clubs, faire une demande pour rejoindre un club, ajouter leur certificat médical, consulter leur panier et leur historique de paiement.
  
  \item \textbf{Adhérents (internes)} : participent aux activités du club, accèdent au calendrier, s’inscrivent et paient en ligne.
  
  \item \textbf{Moniteurs (internes opérationnels)} : consultent les inscrits, contrôlent les certificats et gèrent les événements.
  
  \item \textbf{Comité directeur (administrateurs métier)} : gèrent les utilisateurs, les paiements, les documents administratifs et valident les certificats médicaux.
  
  \item \textbf{Développeur / administrateur technique} : gère la configuration technique, la sécurité, les mises à jour et la maintenance de l’application.
\end{itemize}

\subsection{Confidentialité}

L'application Diving O Club traite des \textbf{données à caractère personnel} dans le cadre
du fonctionnement des clubs FFESSM.  
Elle applique strictement les principes du \textbf{RGPD} : 
minimisation, finalité déterminée et explicite, transparence, limitation de la durée de
conservation, sécurité renforcée et respect des droits utilisateurs 
(\textit{accès, rectification, suppression}).

\subsection*{Exigences de confidentialité et conformité RGPD}

Le projet respecte les obligations de protection des données : 
\textbf{chiffrement}, \textbf{journalisation}, contrôle d’accès (\textbf{RBAC}),
\textbf{chiffrement en transit via HTTPS} et conservation limitée.  
Le tableau ci-dessous récapitule les données traitées dans le MVP, leur finalité, leur base légale
et les mesures de sécurité associées.

\begin{table}[H]
\centering
\small
\setlength{\tabcolsep}{3pt}
\renewcommand{\arraystretch}{1.2}
\begin{tabular}{|p{3cm}|p{4cm}|p{2.8cm}|p{2.5cm}|p{4cm}|}
\hline
\textbf{Données collectées} & \textbf{Finalité} & \textbf{Base légale} & \textbf{Durée de conservation} & \textbf{Mesures de sécurité} \\
\hline

Nom, prénom, email &
Création du compte adhérent ; communication interne du club &
Exécution du contrat / intérêt légitime &
Suppression 12 mois après inactivité ou demande de l’utilisateur &
Stockage chiffré, accès restreint, journalisation des accès \\
\hline

Mot de passe (hashé) &
Authentification et contrôle des accès &
Exécution du contrat &
Suppression immédiate à la suppression du compte &
Hash bcrypt, aucune donnée en clair, politiques de mot de passe fortes \\
\hline

Certificat médical (PDF/JPG) &
Vérification obligatoire de l’aptitude à la pratique &
Obligation réglementaire / sécurité des personnes &
1 an après expiration du certificat ou suppression du compte &
Stockage restreint, accès comité/encadrants uniquement, logs de consultation \\
\hline

Historique d’inscriptions aux événements &
Gestion des activités et traçabilité interne &
Intérêt légitime du club &
3 ans (obligations associatives standard) &
Sécurisation DB, contrôle RBAC, journalisation \\
\hline

Paiements HelloAsso (références, statuts) &
Synchronisation paiement / inscription ; justificatifs comptables &
Obligation comptable / exécution du contrat &
10 ans (obligation comptable) &
Webhook sécurisé, vérification de signature, logs anti-fraude \\
\hline

Logs d’accès (dates, IP approximative, actions) &
Sécurité, détection d’anomalies, audit &
Intérêt légitime (sécurité) &
6 mois (conformément recommandations CNIL) &
Stockage séparé, rotation, analyses d’accès suspects \\
\hline

\end{tabular}
\normalsize
\end{table}

Ces exigences garantissent la conformité du MVP et constituent une base pour le
\textbf{registre des traitements} et la future documentation DPO.

\subsection{Droits d'accès}
\textbf{Droits d'accès :}

\begin{table}[h!]
\centering
\caption{Matrice de permissions par rôle — Diving O Club}
\label{tab:permissions}
\begin{tabular}{|p{6cm}|c|c|c|c|}
\hline
\textbf{Fonctionnalité / Action} & \textbf{Adhérent} & \textbf{Moniteur} & \textbf{Comité directeur} & \textbf{Admin technique} \\ \hline

Créer un compte / se connecter & V & V & V & — \\ \hline
Modifier son profil & V & V & V & — \\ \hline
Voir le calendrier des événements & V & V & V & — \\ \hline
S’inscrire / annuler une inscription & V & V & V & — \\ \hline
Payer via HelloAsso & V & V & V & — \\ \hline
Uploader un certificat médical & V & V & V & — \\ \hline
Valider / refuser certificats & Lecture & Lecture & V (validation) & — \\ \hline
Créer / modifier / supprimer un événement & X & V & V & — \\ \hline
Consulter les participants & V & V & V & — \\ \hline
Gérer les utilisateurs (ajout / rôle / désactivation) & X & X & V & — \\ \hline
Suivi des paiements et statut & X & X & V & — \\ \hline
Configuration du club / documents internes & X & X & V & — \\ \hline
Paramètres techniques / intégrations / logs & X & X & X & V \\ \hline

\end{tabular}
\end{table}

\subsection{Authentification}
\textbf{Système d'authentification :} 

\textbf{Authentification et sécurité :}\\[0.2cm]

\begin{itemize}
  \item \textbf{Connexion} : par email et mot de passe, avec validation des identifiants et contrôle de format (mot de passe conforme à une politique de sécurité via REGEX).
  
  \item \textbf{Sessions} : authentification stateless via JWT d’accès à durée courte, renouvelé par un refresh token stocké en cookie sécurisé (\texttt{httpOnly}). Expiration automatique et révocation en cas de déconnexion ou de réinitialisation.
  
  \item \textbf{Sécurité des comptes} :
  \begin{itemize}
    \item mots de passe hachés (bcrypt)
    \item protection anti-bruteforce
    \item possibilité future d’ajouter une authentification à deux facteurs (2FA)
  \end{itemize}
  
  \item \textbf{Récupération de compte} : procédure par email sécurisé (lien signé à expiration limitée), invalidation des anciennes sessions et limitation de l’utilisation du lien.
  
  \item \textbf{Autorisation par rôle} : accès conditionné par le rôle (adhérent / moniteur / comité directeur), contrôle des droits effectué côté back-end sur chaque route sensible.
\end{itemize}


% =========================
% 2.4 Exigences et choix techniques
% =========================
\section{Exigences et choix techniques}

\subsection{Exigences}

\begin{itemize}
  \item \textbf{Performance :}
  Temps de réponse moyen $< 300$ ms sur les endpoints critiques (authentification, événements, inscriptions, paiements) ; support d’au moins 100 utilisateurs simultanés par club.

  \item \textbf{Sécurité :}
  HTTPS obligatoire, mots de passe hachés (bcrypt), authentification JWT (accès + refresh httpOnly), contrôle d’accès basé sur les rôles (RBAC), journalisation des actions sensibles (certificats, paiements, création/gestion de club).

  \item \textbf{Disponibilité \& résilience :}
  Objectif de 99\% d’uptime en phase bêta, sauvegardes régulières de la base PostgreSQL, reprise possible en cas d’échec de paiement HelloAsso.

  \item \textbf{Scalabilité :}
  Architecture trois-tiers (Front Next.js / Back Nest.js / Base de données), découplée et multi-tenant (\texttt{club\_id}) ; possibilité d’ajouter de nouveaux clubs sans refonte structurelle.

  \item \textbf{Maintenabilité :}
  Code organisé par modules (Nest.js), services, contrôleurs, DTOs, validation côté serveur (Pipes), documentation OpenAPI et pipeline CI/CD avec tests automatisés.

  \item \textbf{Confidentialité \& conformité :}
  Données sensibles chiffrées, contrôle d’accès strict, conformité RGPD (consentement, droit d’accès, droit à l’effacement, limitation des données collectées).
\end{itemize}

\subsection{Choix}
Les choix techniques et leurs justifications objectives sont consignés dans le fichier DECISIONS.md du dépôt. Chaque entrée précise contexte, alternatives et critères de mesure.

\subsection*{Analyse des alternatives techniques}

Au-delà du tableau comparatif ci-dessus, plusieurs solutions alternatives ont été étudiées afin
d’évaluer leur pertinence pour un projet associatif multi-tenant et conforme aux exigences
définies (performance, RGPD, maintenabilité, coûts).

\textbf{Backend / BaaS.}  
Des services managés comme \textbf{Firebase} ou \textbf{Supabase} ont été envisagés pour accélérer le
développement. Ils ont été écartés pour éviter le \textit{lock-in} fournisseur, les coûts variables
d’usage, et les limites de requêtage (Firestore peu adapté aux relations complexes :
inscriptions, certificats, paiements). Nest.js permet au contraire une maîtrise totale de la
structure, des règles de sécurité (guards RBAC) et du schéma multi-tenant.

\textbf{Base de données.}  
\textbf{MySQL} et \textbf{Firestore} ont été analysés comme alternatives à PostgreSQL. MySQL a été
écarté en raison d’un support moins riche pour les contraintes d’intégrité avancées
(nécessaires pour la cohérence événements/inscriptions/paiements). Firestore a été écarté
pour ses limitations en transactions complexes et pour la difficulté à garantir la cohérence
métier d’un club sportif. PostgreSQL s’est imposé pour sa robustesse OLTP et son outillage
mature.

\textbf{Journalisation.}  
La journalisation dans \textbf{PostgreSQL JSONB} a été envisagée mais non retenue, afin de ne pas
mélanger charge métier et volumétrie des logs. Des solutions lourdes comme ELK ont été
écartées par souci de simplicité opérationnelle. \textbf{MongoDB} offre un bon compromis léger,
souple et indépendant.

\textbf{Paiement.}  
Des solutions plus génériques comme \textbf{Stripe} ont été étudiées, mais \textbf{HelloAsso} s’est révélé
plus adapté au contexte associatif français : gestion des campagnes d’adhésion, reçus
automatiques, conformité aux pratiques des clubs FFESSM.

\textbf{Authentification.}  
Des solutions tierces (Auth0, Firebase Auth, Clerk) n’ont pas été retenues pour des raisons de
coût, de dépendance et de contrôle limité sur les jetons. Les \textbf{sessions serveur} ont également
été envisagées mais rejetées au profit d’un modèle \textbf{JWT + refresh httpOnly}, plus simple à
scaler et mieux adapté à un front Next.js.

Cette analyse garantit que les choix retenus ne sont pas arbitraires mais fondés sur une
évaluation comparative cohérente avec les contraintes du projet et du contexte associatif.

\vspace{0.5cm}

\textbf{Frontend — Next.js / React (mobile-first)} \\[0.2cm]
\begin{itemize}
  \item \textbf{Pourquoi Next.js ?} Framework complet, support SSR/SSG si nécessaire, très bon support PWA, performances optimisées, composants réutilisables, accessibilité accrue.
  \item \textbf{Alternatives envisagées :} React seul (moins structuré, moins scalable).
  \item \textbf{Impact exigences :} Parcours mobile-first fluide pour publics non technophiles, interface optimisée, itérations rapides.
\end{itemize}

\vspace{0.3cm}

\textbf{Backend — Node.js / Nest.js} \\[0.2cm]
\begin{itemize}
  \item \textbf{Pourquoi Nest.js ?} Architecture modulaire (modules, services, contrôleurs), fortement typée, maintenable, testable ; logique métier centralisée dans les services ; DTOs et pipes de validation ; guards intégrés pour le RBAC.
  \item \textbf{Alternatives :} Express (plus léger mais moins structuré pour un projet de cette taille).
  \item \textbf{Impact exigences :} Temps de réponse $<300$ms, validation stricte côté serveur, sécurité renforcée, maintenance facilitée.
\end{itemize}

\vspace{0.3cm}

\textbf{Données — PostgreSQL + MongoDB (hybride)} \\[0.2cm]
\begin{itemize}
  \item \textbf{PostgreSQL (données métiers) :} Cohérence transactionnelle (inscriptions, paiements, certificats, clubs, rôles), intégrité référentielle, relations claires.
  \item \textbf{MongoDB (logs / traçabilité) :} Stockage flexible pour journaux, audit trail, rapports et snapshots.
  \item \textbf{Pourquoi deux bases ?} Optimisation selon le type de données : OLTP relationnel (PostgreSQL) + logs/documentation (MongoDB).
  \item \textbf{Impact exigences :} Séparation des charges, meilleure scalabilité multi-club, performance sur les écritures.
\end{itemize}

\subsection*{Comparatif synthétique (alternatives considérées)}
Le tableau ci-dessous présente les choix techniques retenus, les principales alternatives
envisagées, ainsi que la justification de la décision. Ce comparatif répond aux exigences
du référentiel RNCP concernant l’analyse des alternatives.

\begin{table}[H]
\centering
\small
\setlength{\tabcolsep}{4pt}
\renewcommand{\arraystretch}{1.2}
\begin{tabular}{|p{3cm}|p{3cm}|p{5cm}|p{5cm}|}
\hline
\textbf{Choix retenu} & \textbf{Alternative} & \textbf{Justification du choix} & \textbf{Pourquoi l’alternative n’a pas été retenue} \\ \hline

PostgreSQL &
MySQL &
Meilleure gestion des types avancés (JSONB, géodonnées), contraintes fortes et transactions. &
Support plus limité des relations complexes ; moins adapté au futur module « spots de plongée ». \\ \hline

Nest.js (Node) &
Express.js &
Architecture modulaire, sécurisée, testable, inspirée d’Ecto (expérience Phoenix). &
Express requiert beaucoup de « plomberie », pas de structure native, dette technique plus forte. \\ \hline

Next.js (React) &
React seul &
Framework structuré, SSR/ISR, PWA-ready, meilleures performances et SEO. &
Moins structuré, nécessiterait de réimplémenter du routing et de la sécurité. \\ \hline

MongoDB &
Firestore &
Flexible pour les logs, coûts maîtrisés, intégration simple Node.js. &
Coûts variables, verrou fournisseur, limites transactionnelles sur Firestore. \\ \hline

HelloAsso &
Stripe &
Solution gratuite, optimisée pour les associations françaises FFESSM. &
Frais Stripe, complexité comptable, moins aligné avec les pratiques des clubs. \\ \hline

JWT + refresh &
Sessions serveur &
Stateless, scalable, idéal pour Next.js + API Nest. &
Sessions server-side compliquent la scalabilité (sticky sessions). \\ \hline
\end{tabular}
\normalsize
\end{table}


\subsection*{Analyse des alternatives techniques}

Le choix des technologies retenues pour Diving O Club a été précédé d’une
évaluation objective d’alternatives. Le tableau précédent synthétise les options
étudiées, ainsi que les critères qui ont conduit à retenir les solutions finales.
Les décisions ont été motivées par la maintenabilité, la sécurité, la conformité
RGPD, la scalabilité et l’adéquation au contexte associatif.

\paragraph{Frontend : Next.js + React vs React seul}
React seul aurait simplifié l’architecture (SPA), mais l’absence de SSR/ISR,
de routing structuré et d’optimisations intégrées aurait complexifié le
développement long terme. Next.js apporte un cadre plus robuste, une meilleure
performance perçue et une architecture mieux adaptée à un projet évolutif.

\paragraph{Backend : Nest.js vs Express.js}
Express est minimal et rapide à mettre en place, mais impose de structurer
manuellement l’ensemble de l’application (middleware, validation, architecture).
Pour un projet multi-tenant avec logique métier complexe, Nest.js offre un cadre
beaucoup plus sûr et modulaire (DI, modules, pipes, guards), réduisant la dette
technique et les risques de sécurité.

\paragraph{Base de données métier : PostgreSQL vs MySQL}
MySQL a été envisagé, mais PostgreSQL offre des garanties plus fortes en termes
de contraintes relationnelles, de typage, de transactions et de gestion avancée
du JSON (JSONB). Ces caractéristiques sont indispensables pour gérer les relations
clubs $\rightarrow$ événements $\rightarrow$ inscriptions $\rightarrow$ certificats.

\paragraph{Logs et audit : MongoDB vs PostgreSQL JSONB}
Stocker les logs dans PostgreSQL était possible, mais aurait pénalisé les
performances de la base métier. MongoDB permet une grande flexibilité dans la
structure des journaux (schémas variables), un volume important d’écritures et
une séparation claire entre données critiques et données techniques (audit RGPD).

\paragraph{Paiement : HelloAsso vs Stripe}
Stripe est plus complet mais non optimisé pour les associations françaises
(part commissions, justificatifs fiscaux, modèle légal). HelloAsso fournit un
flux parfaitement adapté au contexte associatif, sans frais et avec un
écosystème pensé pour le bénévolat et les clubs.

\paragraph{Authentification : JWT + refresh vs sessions serveur}
Les sessions côté serveur auraient simplifié la logique, mais auraient posé des
contraintes de scalabilité (stockage d’état, sticky sessions). Le couple
JWT + refresh httpOnly permet une architecture stateless plus adaptée au
front-end Next.js, tout en assurant une sécurité renforcée via rotation
des tokens et séparation accès/refresh.


\vspace{0.3cm}

\textbf{Authentification \& sécurité} \\[0.2cm]
\begin{itemize}
  \item JWT accès + refresh cookie \texttt{httpOnly}
  \item RBAC (adhérent / moniteur / comité / admin du club / admin technique)
  \item Hash bcrypt, rate limiting, CORS/TLS, guards Nest.js
  \item Contrôles de validation via DTO + Pipes
  \item \textbf{Pourquoi ?} Standards éprouvés, sécurisés, testables dans CI/CD.
\end{itemize}

\subsection*{Architecture 3-tiers — responsabilités \& flux}
\begin{verbatim}
[Client Web Next.js/React]
     |  HTTPS (JWT accès + cookie refresh httpOnly)
     v
[API Node/Nest.js]
     |-- RBAC, validation DTO/Pipes, logs Mongo
     |-- Intégrations: HelloAsso
     v
[PostgreSQL]  (users, roles, clubs, events, registrations, payments, certificates)
[MongoDB]     (logs, traçabilité, audits, snapshots)
\end{verbatim}

\subsection*{Modèle C4 — Niveau Container (Vue des principaux blocs exécutables)}
\begin{verbatim}



         ------------------------------------------------
        |                                               |
        |                 UTILISATEURS                  |
        -------------------------------------------------
        Adhérent / Moniteur / Membre du comité / Admin technique


                           Navigateur Web
                                   |
                                   v
                +----------------------------------------+
                |      FRONTEND – Next.js / React        |
                |----------------------------------------|
                |  - UI/UX, pages, routing               |
                |  - Formulaires (login, inscription)    |
                |  - Appels API HTTPS (JWT + cookies)    |
                +------------------+---------------------+
                                   |
                                   v
                +----------------------------------------+
                |      BACKEND – API Nest.js             |
                |----------------------------------------|
                |  - Auth (JWT/refresh), RBAC            |
                |  - Clubs, events, inscriptions         |
                |  - Certificats médicaux                |
                |  - Webhooks HelloAsso                  |
                |  - Logs/audit vers MongoDB             |
                |  - Envoi d’emails (SMTP)               |
                +-----------+---------------+-----------+
                            |               |
                            |               |
                          (SQL)           (NoSQL)
                            |               |
                            v               v

         +----------------------+   +------------------------------+
         | PostgreSQL           |   | MongoDB                      |
         |----------------------|   |------------------------------|
         | - Users, Roles       |   | - Logs applicatifs           |
         | - Clubs, Events      |   | - Audits, traçabilité        |
         | - Registrations      |   | - Erreurs                    |
         | - Certificates       |   +------------------------------+
         +----------------------+

                            ^
                            |
                 Webhooks / Statuts paiements
                            |
                            v

                +----------------------------------------+
                |          HelloAsso API                 |
                |----------------------------------------|
                | - Gestion paiements associatifs        |
                | - Renvoi statuts → webhooks API        |
                +----------------------------------------+


\end{verbatim}


\vspace{0.5cm}

\textbf{Paiements — HelloAsso} \\[0.2cm]
\begin{itemize}
  \item \textbf{Pourquoi ?} Solution dédiée aux associations françaises, reçus automatiques, workflow paiements simple.
  \item \textbf{Alternatives :} Stripe (plus générique, moins adapté aux clubs FFESSM).
\end{itemize}

\vspace{0.3cm}

\textbf{Scalabilité \& évolutivité} \\[0.2cm]
\begin{itemize}
  \item Multi-tenant via \texttt{club\_id}
  \item Cache applicatif (post-MVP)
  \item Extensions prévues : 2FA, carnet de suivi, boutique plus complète, évolution Next.js SSR
\end{itemize}

\vspace{0.3cm}

\textbf{Maintenabilité \& qualité} \\[0.2cm]
\begin{itemize}
  \item Modules Nest.js (users, auth, clubs, events, payments, certificates)
  \item Tests API des endpoints critiques
  \item Linting / formatage / documentation OpenAPI
  \item Pipeline CI/CD (tests $\rightarrow$ build $\rightarrow$ déploiement)
\end{itemize}

% =========================
% 2.5 Définition du MVP
% =========================
\section{Définition du MVP}

\subsection*{Méthode de priorisation MoSCoW}

Afin de définir le périmètre fonctionnel prioritaire, la méthode MoSCoW a été utilisée. 
Cette approche, couramment utilisée en gestion de projet agile, permet de classer les fonctionnalités selon leur niveau de criticité pour la première version du produit.

Les priorités ont été établies en tenant compte :
\begin{itemize}
    \item des besoins réels identifiés auprès du club de plongée (entretiens avec le président, les moniteurs et les adhérents)
    \item des contraintes associatives (bénévolat, temps limité, simplicité d'usage)
    \item des objectifs métier (réduction de la charge administrative, fiabilité du suivi)
    \item des capacités de développement dans le cadre du projet CDA (une seule ressource, planning défini)
\end{itemize}

Cette analyse garantit que la première version (MVP) se concentre uniquement sur les fonctionnalités indispensables à l’utilisation opérationnelle du système en conditions réelles, à savoir créer un compte, se connecter, rechercher un club et afficher la page du club.

\vspace{0.5cm}

\begin{center}
{\Large \textbf{Must Have}}
\end{center}

\vspace{0.2cm}
\begin{tabularx}{\textwidth}{|X|X|}
\hline
\textbf{Fonctionnalité} & \textbf{Justification / Raison du classement} \\ \hline

Authentification et rôles (adhérent / moniteur / comité) &
Condition indispensable pour accéder aux données personnalisées et au multi-tenant. Bloque toutes les autres fonctionnalités. \\ \hline

Création de club par un utilisateur externe & Prérequis fondamental : sans club créé, aucune gestion d’adhérents, d’événements ou de certificats n’est possible. \\ \hline

Gestion des adhérents &
Fonction coeur du projet. Sans elle, aucun suivi ni inscription possible. Valeur utilisateur immédiate. \\ \hline

Gestion des événements (création, édition, calendrier) &
Fonction essentielle au fonctionnement du club : toute l'organisation repose sur ce module. \\ \hline

Inscriptions aux événements &
Indispensable pour permettre aux adhérents de participer aux activités. Forte valeur utilisateur. \\ \hline

Paiements avec HelloAsso &
Nécessaire pour automatiser les activités payantes et réduire les erreurs de gestion. \\ \hline

Gestion des certificats médicaux (upload + validation) &
Exigence légale pour la pratique de la plongée. Priorité réglementaire incontournable. \\ \hline

\end{tabularx}


\begin{center}
{\Large \textbf{Should Have}}
\end{center}

\vspace{0.2cm}

\begin{tabularx}{\textwidth}{|X|X|}
\hline
\textbf{Fonctionnalité} & \textbf{Justification / Raison du classement} \\ \hline

Emails / notifications automatiques &
Réduction des relances manuelles. Améliore le confort du comité mais non bloquant pour la V1. \\ \hline

Tableau de bord simplifié (suivi inscriptions / paiements / validations) &
Apporte de la visibilité et facilite la gestion interne, mais le fonctionnement de base reste possible sans. \\ \hline

\end{tabularx}

\begin{center}
{\Large \textbf{Could Have}}
\end{center}

\vspace{0.2cm}

\begin{tabularx}{\textwidth}{|X|X|}
\hline
\textbf{Fonctionnalité} & \textbf{Justification / Raison du classement} \\ \hline

Gestion du matériel du club &
Fonction utile mais non critique pour le fonctionnement administratifs des activités. Peut être reportée. \\ \hline

Carnet de suivi numérique &
Valeur ajoutée pour les adhérents mais hors périmètre prioritaire du CDA et du MVP. \\ \hline

\end{tabularx}

\begin{center}
{\Large \textbf{Won't Have}}
\end{center}

\vspace{0.2cm}

\begin{tabularx}{\textwidth}{|X|X|}
\hline
\textbf{Fonctionnalité} & \textbf{Justification / Raison du classement} \\ \hline

Statistiques avancées et exports PDF &
Fonctionnalités lourdes et non indispensables pour la V1. Prévu en phase 2. \\ \hline

Intégration FFESSM (structures régionales / départementales) &
Hypothèse de travail intéressante mais pas réalisable dans la première version. \\ \hline

\end{tabularx}

\vspace{0.5cm}

\textbf{Définition du MVP :} \\

\textbf{MVP — Avril 2026}\\[0.2cm]

Le MVP correspond à la première version pleinement fonctionnelle permettant à un club
d’utiliser l’application en conditions réelles. Il inclut : l’authentification et les rôles,
la création de compte, la \textbf{création de club sur demande} (avec \textbf{validation manuelle} et
attribution initiale des droits au créateur), la gestion des événements, les inscriptions
(\textbf{avec ajout automatique au panier}), la synchronisation des paiements via HelloAsso, ainsi
que la gestion de la conformité (\textit{dépôt et validation du certificat médical}).  
Ce périmètre couvre le module cœur indispensable avant l’intégration du front, des paiements réels et des optimisations prévues entre mai et juin 2026.

\vspace{0.4cm}

\textbf{Scénarios essentiels du MVP :}\\[0.2cm]

\begin{enumerate}
  \item Connexion utilisateur et gestion de session (accès sécurisé par rôle : adhérent / moniteur / comité)
  \item Consultation du calendrier des événements (liste, détails, capacité, conditions d’accès)
  \item Inscription et annulation à un événement (blocage si certificat absent ou expiré)
  \item Paiement via HelloAsso pour les événements payants (statut de paiement synchronisé côté back-end)
  \item Gestion administrative de base côté comité :
  \begin{itemize}
      \item création / modification d’un événement
      \item validation ou refus d’un certificat médical
      \item suivi des inscrits et statuts de paiement
  \end{itemize}
\end{enumerate}

\vspace{0.4cm}

\textbf{Critères de succès du MVP :}\\[0.2cm]

\begin{itemize}
  \item L’ensemble des scénarios listés est utilisable sans support par au moins 20 membres du club testeur
  \item Réduction mesurée d’au moins 50\% des relances administratives manuelles habituelles
\end{itemize}

\subsection*{KPIs produit (mesure d'impact)}
\begin{itemize}
  \item \textbf{Réduction des relances admin} : $-50\%$ (moyenne hebdo, comité)
  \item \textbf{Adoption bêta} : $\geq 20$ utilisateurs actifs/mois (club pilote)
  \item \textbf{Temps moyen d'inscription} : $< 2$ min (du clic à confirmation)
  \item \textbf{Taux d’abandon paiement} : $< 15\%$ sur événements payants
  \item \textbf{Taux d’erreur certificat} : $< 5\%$ (uploads invalides/refusés)
  \item \textbf{Perf endpoints critiques} : $p_{95} < 300$ ms (auth, events, register, payment)
\end{itemize}


\subsection*{Scénarios essentiels \& critères d'acceptation (MVP)}

Les critères d’acceptation du projet sont rédigés en \textbf{syntaxe Gherkin} 
(\textit{Given / When / Then}), afin d’être 
\textbf{vérifiables, non ambigus et testables}.

\begin{itemize}
  \item de décrire clairement les comportements attendus ;
  \item de garantir une compréhension partagée entre les parties prenantes ;
  \item d’établir des critères d’acceptation objectifs et mesurables ;
  \item de faciliter l’automatisation future des tests fonctionnels.
\end{itemize}

\subsection*{Méthode de rédaction des critères d'acceptation}

Chaque User Story du projet possède des critères d’acceptation rédigés en 
\textbf{syntaxe Gherkin (Given / When / Then)} et documentés dans les 
\textbf{Issues du GitHub Project}.  
Pour ne pas alourdir ce document, quelques exemples représentatifs sont présentés ci-dessous.


\vspace{0.5cm}

\subsection*{US-01 — Inscription (register) — Critères d'acceptation (Gherkin)}

\begin{verbatim}
Scenario: Inscription réussie
  Given un utilisateur non inscrit
  When il poste POST /auth/register avec name, un email valide 
       et un mot de passe conforme à la politique
  Then l’API répond 201 Created avec un corps JSON contenant userId, email, name (sans password)
  And un enregistrement User existe en base avec email_verified = false 
       et created_at non nul
  And un email de vérification est envoyé à l’adresse fournie 
       (événement email.verification.requested loggé)
  And aucune session persistante n’est créée tant que l’email n’est pas vérifié 
       (pas de cookie refresh)

Scenario: Email déjà utilisé
  Given un utilisateur non inscrit
  When il poste POST /auth/register avec un email déjà existant
  Then l’API répond 409 Conflict avec code = "EMAIL_TAKEN"
  And aucun nouvel utilisateur n’est créé en base
  And aucun email de vérification n’est envoyé

Scenario: Mot de passe invalide
  Given un utilisateur non inscrit
  When il poste POST /auth/register avec un mot de passe ne respectant pas la politique 
  Then l’API répond 400 Bad Request avec code = "WEAK_PASSWORD" 
       et la liste des règles violées
  And aucun utilisateur n’est créé
  And aucune session ni email n’est généré
\end{verbatim}

\subsection*{Plan de validation utilisateur du MVP}

Afin de garantir que le MVP r\'epond aux besoins r\'eels du club et des utilisateurs 
finaux, une session de validation sera organis\'ee avec un panel repr\'esentatif 
(adh\'erents, moniteurs, membres du comit\'e). Cette \'etape permet de tester 
l’ergonomie, la compr\'ehension des parcours, la coh\'erence des informations 
et la fiabilit\'e des fonctionnalit\'es essentielles.

\subsubsection*{Objectifs de la session de test}
\begin{itemize}
    \item Valider que les parcours principaux (connexion, inscription, certificat, paiement) sont compr\'ehensibles sans assistance.
    \item Identifier les frictions, incompr\'ehensions ou lenteurs dans l’usage r\'eel.
    \item Recueillir les retours des utilisateurs afin d’am\'eliorer la version Beta.
    \item Confirmer que le MVP est utilisable en conditions de club.
\end{itemize}

\subsubsection*{Panel utilisateur}
\begin{itemize}
    \item \textbf{3 adh\'erents} dont 1 peu technophile.
    \item \textbf{2 moniteurs / encadrants}.
    \item \textbf{1 membre du comit\'e} (tr\'esori\`ere ou secr\'etaire).
\end{itemize}

\subsubsection*{Fonctionnalit\'es test\'ees}
\begin{itemize}
    \item Connexion / cr\'eation de compte.
    \item Consultation du calendrier des plong\'ees.
    \item Inscription \`a un \'ev\'enement.
    \item Upload et validation du certificat m\'edical.
    \item Paiement HelloAsso et synchronisation du statut.
    \item Acc\`es au tableau de bord utilisateur.
\end{itemize}

\subsubsection*{Scénarios de validation}
Les tests suivront un protocole bas\'e sur des scénarios réels :

\begin{itemize}
    \item \textbf{T1} : Un adh\'erent s’inscrit \`a une plong\'ee, v\'erifie la place restante et valide son certificat.
    \item \textbf{T2} : Un moniteur consulte les inscrits, v\'erifie les certificats et contr\^ole les paiements.
    \item \textbf{T3} : Un adh\'erent peu technophile tente de naviguer seul sur le site (mobile-first).
    \item \textbf{T4} : Un membre du comit\'e utilise l’interface d’administration minimale.
\end{itemize}

\subsubsection*{M\'ethode de recueil des feedbacks}
\begin{itemize}
    \item Observation directe (dur\'ee, erreurs, blocages).
    \item Questionnaire de satisfaction court (5 questions fermées + 1 ouverte).
    \item Notation de l’effort perçu (\emph{User Effort Score}).
    \item Compte rendu synthétique intégré dans le GitHub Project (tickets d’amélioration).
\end{itemize}

\subsubsection*{Livrables attendus}
\begin{itemize}
    \item Fiche de test complétée pour chaque utilisateur.
    \item Synthèse des problèmes rencontrés.
    \item Liste des actions correctives prioritaires (MoSCoW).
    \item Ajout des améliorations dans le backlog Beta.
\end{itemize}

\subsubsection*{Calendrier prévisionnel}
\begin{itemize}
    \item Semaine 1 : Préparation des scénarios et du protocole.
    \item Semaine 2 : Session de test (2h, club pilote).
    \item Semaine 3 : Analyse des retours et intégration des correctifs dans la roadmap Beta.
\end{itemize}

\subsection*{Revue et validation des exigences avec les utilisateurs métier}

Dans le cadre du projet, une première \textbf{validation réelle des exigences} a été réalisée
avec les parties prenantes du club pilote (Aquaclub21).  
Lors de la réunion du comité directeur du \textbf{15 février 2025}, les besoins métier suivants ont été
confirmés et priorisés :

\begin{itemize}
  \item centralisation certificats–paiements–inscriptions ;
  \item simplification du parcours d’inscription adhérent ;
  \item visibilité immédiate des inscrits pour les encadrants ;
  \item importance du mobile-first pour les adhérents peu technophiles.
\end{itemize}

Un \textbf{compte rendu de réunion} (PDF) ainsi qu’une mise à jour du \textbf{GitHub Project}
(section \textit{Issues $\rightarrow$ Validation métier}) assurent la traçabilité de cette validation.

\vspace{0.5cm}
Afin de garantir que les fonctionnalités du MVP correspondent fidèlement aux besoins 
du club, une revue d’exigences sera organisée avec les parties prenantes métier 
(comité directeur, moniteurs, adhérents représentatifs). Cette revue permet de 
valider collectivement le périmètre, les priorités et les critères d’acceptation 
avant le développement.

\subsubsection*{Objectifs de la revue}
\begin{itemize}
    \item Vérifier que les exigences fonctionnelles couvrent bien les besoins du club.
    \item Identifier les oublis, ambiguïtés ou conflits entre exigences.
    \item Aligner les priorités (MoSCoW) avec les utilisateurs métier.
    \item Confirmer la cohérence entre le MVP, les user stories et les parcours utilisateurs.
\end{itemize}

\subsubsection*{Participants}
\begin{itemize}
    \item 1 à 2 membres du comité (président / trésorière).
    \item 1 moniteur encadrant (usage opérationnel terrain).
    \item 1 adhérent représentatif (peu technophile).
    \item Développeur / porteur du projet (facilitateur).
\end{itemize}

\subsubsection*{Déroulé de la session (1h30)}
\begin{itemize}
    \item Présentation synthétique du périmètre MVP.
    \item Lecture collaborative des user stories clés (US-01 à US-08).
    \item Vérification des critères d’acceptation (Given / When / Then).
    \item Revue de la matrice MoSCoW pour ajuster les priorités.
    \item Collecte des retours (questions, frustrations, risques perçus).
    \item Mise à jour du backlog sur GitHub Project.
\end{itemize}

\subsubsection*{Livrables produits}
\begin{itemize}
    \item Compte rendu de revue (PDF) : décisions, modifications, arbitrages.
    \item Mise à jour du backlog (priorités revues et exigences clarifiées).
    \item Liste des points ouverts / questions à trancher pour la Beta.
    \item Validation formelle des parties prenantes (signature numérique ou confirmation écrite).
\end{itemize}

\subsubsection*{Calendrier prévisionnel}
\begin{itemize}
    \item \textbf{J+0} : Envoi des exigences et des user stories aux participants.
    \item \textbf{J+7} : Session de revue (club pilote).
    \item \textbf{J+10} : Intégration des retours dans le backlog GitHub.
\end{itemize}


\paragraph{Bornes techniques minimales (V1)}


\begin{itemize}
  \item \textbf{Modèles V1 uniquement} : Users, Roles, Events, Registrations, Payments, MedicalCertificates.
  \item \textbf{Paiement} : parcours carte via HelloAsso (pas de remboursement ni d'export comptable).
  \item \textbf{Sécurité} : JWT accès court + refresh cookie httpOnly, RBAC simple (3 rôles).
  \item \textbf{Journalisation} : logs d'actions sensibles (validation certificat, création d'événement).
  \item \textbf{Admin} : back-office minimal (liste utilisateurs, événements, validations).
\end{itemize}

\begin{center}
\textbf{Utilisateur} $\rightarrow$ \textbf{Connexion} $\rightarrow$ \textbf{Inscription} $\rightarrow$ 
\textbf{Certificat} $\rightarrow$ \textbf{Paiement} $\rightarrow$ \textbf{Confirmation}
\end{center}

\textbf{Diagramme — MVP (Avril 2026) :}
\begin{verbatim}
                  +==============================================+
                  |            MVP — Avril 2026              |
                  | Module cœur : Événements / Paiements / Auth  |
                  +==============================================+
                                      |
                     +----------------+----------------+
                     |                |                |
                     v                v                v
            +-----------+   +-------------+   +---------------+   
            | Auth &    |   | Calendrier  |   | Inscriptions  |  
            | rôles     |   | événements  |   | événements    |  
            | (Must)    |   | (Must)      |   | (Must)        |  
            +-----------+   +-------------+   +------+--------+   
                                                          |         
                                                          v         
                                       +------------------------------+
                                       | Certificats médicaux         |
                                       | Upload + validation (Must)   |
                                       +------------------------------+
                                                          |
                                                          v
                                           +---------------------------+
                                           | Paiements HelloAsso      |
                                           | (Must)                   |
                                           +---------------------------+
\end{verbatim}

\textbf{Diagramme MVP v1.1 (post-V1) :}

\begin{verbatim}
                  +==============================================+
                  |           Post-MVP — Septembre 2026          |
                  |        Dashboard / Notifications             |
                  +==============================================+
                                      |
                     +----------------+-----------------------+
                     |                |                       |
                     v                v                       v
            +-----------+   +-------------+     +-------------------------+
            | Dashboard |   | Notifications|     | Gestion matériel club  |
            | (Should)  |   | (Should)     |     | (Could)                |
            +-----------+   +-------------+     +-----------+-------------+
                                                               |
                                                               v
                                                      +--------------------+
                                                      | Carnet de suivi    |
                                                      | (Could)            |
                                                      +--------------------+
\end{verbatim}

\subsection*{Plan de validation avec utilisateurs (club pilote)}
\begin{itemize}
  \item \textbf{Ateliers cadrage (Semaine 1)} : 1h avec comité (priorités), 30min avec 2 moniteurs (contraintes terrain)
  \item \textbf{Tests exploratoires (mensuels)} : 5 adhérents + 1 moniteur ; scénarios MVP ; recueil friction
  \item \textbf{Canaux feedback} : formulaire court après inscription/paiement ; salon Discord dédié
  \item \textbf{Revue bimensuelle} : démonstration au comité, décisions de priorisation (MoSCoW)
  \item \textbf{Critères de sortie bêta} : KPIs atteints 2 semaines consécutives ; zéro bug bloquant
\end{itemize}


\subsection*{Analyse des risques (probabilité / impact / plan d'action)}
\begin{table}[h!]
\centering
\begin{tabular}{|p{3.5cm}|p{2.2cm}|p{2.2cm}|p{5.5cm}|p{2.5cm}|}
\hline
\textbf{Risque} & \textbf{Prob.} & \textbf{Impact} & \textbf{Mitigation} \\ \hline
Charge projet sous-estimée & Moyenne & Élevé & Scope strict MVP, jalons bimensuels, coupe des « Should » si dérive \\ \hline
Adoption faible (adhérents) & Moyenne & Moyen & Mobile-first, tutoriels, tests mensuels, simplification parcours \\ \hline
HelloAsso indisponible & Faible & Élevé & Mode secours: inscription « non payée » + validation manuelle \\ \hline
Bug critique prod & Faible & Élevé & Tests API CI, rollback rapide, sauvegardes quotidiennes \\ \hline
Perte de données & Très faible & Très Élevé & Backups PostgreSQL, restauration testée mensuellement, logs Mongo \\ \hline
RGPD (accès non autorisé) & Faible & Élevé & RBAC strict, logs d'accès, revue des permissions \\ \hline
\end{tabular}
\end{table}

\begin{flushleft}
\subsection*{Matrice des risques (Probabilit\'e $\times$ Impact)}
\small
\begin{tabular}{p{2.2cm}
                >{\centering\arraybackslash}p{1.1cm}
                >{\centering\arraybackslash}p{1.1cm}
                >{\centering\arraybackslash}p{1.2cm}
                p{3.4cm}
                p{3.4cm}
                p{2.4cm}}
\toprule
\textbf{Risque} & \textbf{Prob. (1--5)} & \textbf{Impact (1--5)} & \textbf{Score} &
\textbf{Mitigation (pr\'eventif)} & \textbf{Plan de contingence} & \textbf{Indicateur/Seuil} \\
\midrule

Charge projet sous-estim\'ee & 3 & 4 & 12 &
Scope strict MVP, revues bimensuelles, couper les \og Should \fg{} si d\'erive &
D\'eprioriser non-critiques, extension d\'elai valid\'ee, renfort ponctuel &
Burndown \textgreater{} 20\% sur 2 sprints \\

\addlinespace

Adoption faible (adh\'erents) & 3 & 3 & 9 &
Mobile-first, parcours simplifi\'e, tutoriels, tests mensuels &
Campagne d’onboarding cibl\'ee, accompagnement club pilote &
\textless{} 20 utilisateurs actifs/mois ou churn \textgreater{} 30\% \\

\addlinespace

Indisponibilit\'e HelloAsso & 2 & 4 & 8 &
Timeouts, retries, surveillance des 5xx, d\'egradation contr\^ol\'ee &
Mode \og inscription non pay\'ee \fg{} + validation manuelle; relance paiement &
Taux 5xx \textgreater{} 2\% ou latence \textgreater{} 5s (p95) \\

\addlinespace

Bug critique en prod & 2 & 4 & 8 &
CI tests API, feature flags, Sentry/Logs &
Rollback rapide, hotfix, correctifs prioris\'es &
\textgreater{} 5 erreurs critiques/jour (Sentry) \\

\addlinespace

Perte de donn\'ees & 1 & 5 & 5 &
Backups quotidiens, test de restauration mensuel, politique de r\'etention &
Restauration \`a J-1, bascule instance r\'eplica &
Backup KO \textgreater{} 24h ou test restore \'echou\'e \\

\addlinespace

RGPD / acc\`es non autoris\'e & 2 & 4 & 8 &
RBAC strict, journaux d’acc\`es, revues p\'eriodiques, chiffrement &
D\'econnexion globale, rotation de cl\'es, notification incident &
Alarmes sur acc\`es anormaux \\

\bottomrule
\end{tabular}
\normalsize
\end{flushleft}


\vspace{0.5cm}


\subsection*{Carte chaleur Probabilit\'e $\times$ Impact}
\setlength{\tabcolsep}{6pt}
\renewcommand{\arraystretch}{1.2}
\begin{tabular}{c|*{5}{>{\centering\arraybackslash}p{1.6cm}}}
\multicolumn{1}{c}{} & \multicolumn{5}{c}{\textbf{Impact}}\\
\textbf{Prob.} & \textbf{1} & \textbf{2} & \textbf{3} & \textbf{4} & \textbf{5}\\\hline
\textbf{5} & \cellcolor{riskMed}5 & \cellcolor{riskMed}10 & \cellcolor{riskHigh}15 & \cellcolor{riskHigh}20 & \cellcolor{riskHigh}25\\
\textbf{4} & \cellcolor{riskLow}4 & \cellcolor{riskMed}8 & \cellcolor{riskMed}12 & \cellcolor{riskHigh}16 & \cellcolor{riskHigh}20\\
\textbf{3} & \cellcolor{riskLow}3 & \cellcolor{riskLow}6 & \cellcolor{riskMed}9 & \cellcolor{riskHigh}12 & \cellcolor{riskHigh}15\\
\textbf{2} & \cellcolor{riskLow}2 & \cellcolor{riskLow}4 & \cellcolor{riskLow}6 & \cellcolor{riskMed}8 & \cellcolor{riskMed}10\\
\textbf{1} & \cellcolor{riskLow}1 & \cellcolor{riskLow}2 & \cellcolor{riskLow}3 & \cellcolor{riskLow}4 & \cellcolor{riskMed}5\\
\end{tabular}



\clearpage

% =========================
% 2.6 Roadmap
% =========================
\section{Roadmap}

La roadmap ci-dessous reprend les jalons majeurs du projet, en cohérence avec le planning général présenté au Chapitre 1. Elle distingue les versions prévues dans le cadre du diplôme (jusqu’à la V1), puis les évolutions envisagées après soutenance.

\vspace{0.4cm}

Les jalons officiels du projet sont définis dans les \textbf{GitHub Milestones}, qui servent
de référence pour le pilotage, la planification et le suivi des issues. La roadmap présentée
dans ce document s'aligne strictement sur ces milestones.

\begin{table}[H]
\centering
\small
\setlength{\tabcolsep}{4pt}
\renewcommand{\arraystretch}{1.2}
\begin{tabular}{|p{4cm}|p{3cm}|p{8cm}|}
\hline
\textbf{Milestone GitHub} & \textbf{Échéance} & \textbf{Contenu / livrables} \\ \hline

\textbf{v0.1 — Conception et cadrage} (Sept–Dec 2024) &
27 décembre 2025 &
Définition de l’architecture, du périmètre fonctionnel, diagrammes,
MCD/MLD, cahier des charges, premières user stories. \\ \hline

\textbf{v0.2 — PoC Release} (Jan 2025) &
17 janvier 2026 &
Prototype reliant front–back–API–base de données, validation technique
de la stack (Next.js, Nest.js, PostgreSQL, MongoDB). \\ \hline

\textbf{v0.3 — MVP Release} (Fév 2025) &
14 février 2026 &
Première version fonctionnelle : inscription, connexion, consultation
des clubs et des informations essentielles. \\ \hline

\textbf{v0.4 — Beta Release} (Avr 2025) &
25 avril 2026 &
Version testable par le club pilote : gestion des membres, événements,
profil utilisateur, parcours mobile-first. \\ \hline

\textbf{v1.0 — Launch} (Jun 2025) &
7 juin 2026 &
Version complète multi-clubs : rôles, paiements HelloAsso, certificats,
tableau de bord simplifié, stabilisation et correctifs. \\ \hline

\end{tabular}
\normalsize
\end{table}

Toutes les issues de développement sont associées à ces milestones afin d’assurer une
traçabilité complète et un pilotage rigoureux. Chaque milestone contient plusieurs
\textbf{epics}, eux-mêmes composés de \textbf{user stories} clairement définies.  

Chaque user story comporte une liste de tâches sous forme de \textbf{checkboxes} permettant
de suivre l’avancement opérationnel. Pour chaque user story, une \textbf{branche Git dédiée}
est créée afin d’isoler le développement.  

Une fois les tâches terminées, la branche est poussée en incluant la référence de l’issue
dans le message de commit, ce qui permet de \textbf{clôturer automatiquement l’issue}
(\#numéro).  

Dans le cadre d’un projet individuel, l’usage systématique des Pull Requests (PR) serait
redondant et chronophage : elles ne sont donc pas utilisées. L’objectif est de clôturer
l’ensemble des user stories afin de clôturer les epics associés, puis de valider la milestone
qui les regroupe.

\vspace{0.4cm}

\textbf{Évolutions post-diplôme (vision produit)}\\[-0.1cm]
\begin{itemize}
  \item Notifications automatiques (paiement, certificat expiré)
  \item Tableau de bord administratif complet
  \item Gestion du matériel du club
  \item Carnet de suivi numérique
  \item Exports avancés (CSV / PDF)
  \item Montée en échelle (multi-clubs optimisé)
  \item Intégration FFESSM (structures départementales / régionales)
  \item Sécurité renforcée (2FA comité, audit avancé)
\end{itemize}


% =========================
% 2.7 Liens utiles
% =========================
\section{Liens utiles}

\begin{itemize}
    \item User Stories: \url{https://www.mountaingoatsoftware.com/agile/user-stories}
    \item MoSCoW: \url{https://www.productplan.com/glossary/moscow-prioritization/}
    \item PostgreSQL Docs: \url{https://www.postgresql.org/docs/}
    \item MongoDB Modeling: \url{https://bit.ly/mongodb-modeling}
    \item Architecture 3-tier: \url{https://en.wikipedia.org/wiki/Multitier_architecture}
\end{itemize}
